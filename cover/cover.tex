\documentclass[12pt]{article}

% KNOWN PROBLEMS -- TODO
%
% - text positions are not safe against dimension changes; easy to fix.
% - the wave will have a slightly different angle on the right if the bleed width changes; hard to fix. Not worth the effort.

\usepackage{graphicx}

% My guess for the bad colors I had on a first test print: use of the RGB color space.
% Let's try CMYK
% http://tex.stackexchange.com/questions/125028/xelatex-and-tikz-how-do-i-ensure-my-pdfs-are-using-the-cmyk-color-model

% A sans serif font that I want to try.
% Switch to this font with the \fcofamily command.
\usepackage{comfortaa}

% Filename of the UGent logo
% Choose eps version for high quality.
\newcommand{\LogoFile}{translogoneg}

\newcommand{\MyBindingTitle}{Towards Flexible Goal-Oriented Logic Programming}
% \MyFrontCoverTitle may contain linebreaks (\\). The others may not.
\newcommand{\MyFrontCoverTitle}{Towards\\Flexible Goal-Oriented Logic Programming}

\newcommand{\MyBindingName}{B.~Desouter}
\newcommand{\MyFrontCoverName}{Benoit Desouter}

% Important length settings
% =========================
%
% The relevant lengths are defined in the following TUG Boat article: http://tug.org/pracjourn/2007-1/robbers/robbers.pdf
%
%       (a')                  (e')  (f') (g')  (h')                   (b')
% +-----+------------------------+--+-------+--+-------------------------+-----+
% |<. pdf paper width .................................................^......>|
% |                                                                    B       |
% |     (a)                      (e)(f)     (g)(h)                     v (b)   |
% |     +------------------------+--+-------+--+-------------------------+     |
% |     |                        |  |       |  |                         |     |
% |<.A.>|                        | b|<. C .>| b|                         |<.A.>|
% |     |                        | i|       | i|                         |     |
% |     |<. book paper width ...>| n|       | n|<. book paper width ....>|     |
% |     |                        | d|       | d|                         |     |
% |     |                        | i|       | i|                         |     |
% |     |                        | n|       | n|                         |     |
% |     |                        | g|       | g|                         |     |
% |     |                        |  |       |  |                         |     |
% |     |                        | c|       | c|                         |     |
% |     |                        | o|       | o|                         |     |
% |     |                        | r|       | r|                         |     |
% |     |                        | r|       | r|                         |     |
% |     |                        | e|       | e|                         |     |
% |     |                        | c|       | c|                         |     |
% |     |                        | t|       | t|                         |     |
% |     |                        | i|       | i|                         |     |
% |     |                        | o|       | o|                         |     |
% |     |                        | n|       | n|                         |     |
% |     |                        |  |       |  |                         |     |
% |     |                        |  |       |  |                         |     |
% |     |                        |  |       |  |                         |     |
% |     |                        |  |       |  |                         |     |
% |     |                        |  |       |  |                         |     |
% |     |                        |  |       |  |                         |     |
% |     |                        |  |       |  |                         |     |
% |     |                        |  |       |  |                         |     |
% |     |                        |  |       |  |                         |     |
% |     +------------------------+--+-------+--+-------------------------+     |
% |     (c)                      (i)(j)     (k)(l)                     ^ (d)   |
% |                                                                    B       |
% |                                                                    v       |
% +-----+------------------------+--+-------+--+-------------------------+-----+
%       (c')                  (i')  (j') (k')  (l')                   (d')
%
% where A := bleed width
% where B := bleed height
% where C := binding width
% where binding correction has a width := binding correction width
%
% where (a) := (BackCoverTopLeft)
% where (b) := (FrontCoverTopRight)
% where (c) := (BackCoverBottomLeft)
% where (d) := (FrontCoverBottomRight)
% where (e) := (BackCoverTopRight)
% where (f) := (BindingTopLeft)
% where (g) := (BindingTopRight)
% where (h) := (FrontCoverTopLeft)
% where (i) := (BackCoverBottomRight)
% where (j) := (BindingBottomLeft)
% where (k) := (BindingBottomRight)
% where (l) := (FrontCoverBottomLeft)
%
% But perhaps the points (a') := (aPrime) to (l') := (lPrime) are even more important.
%
% The pdf paper width is not equal to  2 x (book paper width + binding correction width) + binding width because printers cannot print near the border. Therefore one uses bleed that is cut away (approximately).
% Print the cover on a physical sheet of at least pdf paper width x pdf paper height.

% Declaring the relevant lengths
\newlength{\MyPdfPaperWidth}
\newlength{\MyPdfPaperHeight}
\newlength{\MyBleedWidth}
\newlength{\MyBleedHeight}
\newlength{\MyBookPaperWidth}
\newlength{\MyBookPaperHeight}
\newlength{\MyBindingCorrectionWidth}
\newlength{\MyBindingWidth}

% Set the lengths
% You can adapt these at the last minute, I have defined everything in terms of those so that the overall appearance won't change (hopefully).
\setlength{\MyBleedWidth}{3mm}
\setlength{\MyBleedHeight}{3mm}
\setlength{\MyBookPaperWidth}{170mm}
\setlength{\MyBookPaperHeight}{240mm}
\setlength{\MyBindingCorrectionWidth}{3mm}
\setlength{\MyBindingWidth}{14mm} % The binding width of Karel's PhD thesis was 18mm, for 270 pages University Press says 14mm (see their website and emails).

% Allow calculations with lengths
\usepackage{calc}

% Set the derived lengths
% DO NOT CHANGE
% Remember that for multiplication with a scalar, the scalar must be the second factor.
\setlength{\MyPdfPaperWidth}{(\MyBleedWidth + \MyBookPaperWidth + \MyBindingCorrectionWidth)*2 + \MyBindingWidth}
\setlength{\MyPdfPaperHeight}{\MyBleedHeight * 2 + \MyBookPaperHeight}

% Set the pdf page dimensions
\usepackage[paperwidth=\MyPdfPaperWidth,paperheight=\MyPdfPaperHeight]{geometry}

\usepackage{tikz}
\usetikzlibrary{backgrounds, calc, intersections}

\pgfdeclarelayer{background}
\pgfdeclarelayer{circles}
\pgfdeclarelayer{foreground}
\pgfdeclarelayer{auxiliaryLines}
\pgfsetlayers{background,circles,main,foreground,auxiliaryLines}

% Do not use RGB colors, I suppose.
% Use http://www.rapidtables.com/convert/color/rgb-to-cmyk.htm
% % The same blue color scheme as the example on which I based this code.
% % Unexpectedly, it is rather nice.
% % But many people have a blue thesis cover.
% % I also want to try shades of dark green.
% \definecolor{ColorScheme1Blue}{RGB}{81,91,199} % Meant for front and back cover background.
% \definecolor{ColorScheme1DarkBlue}{RGB}{66,76,103} % Meant for binding background.
% \definecolor{ColorScheme1Gray}{RGB}{112,121,139} % Meant for wave that is mostly covered and for the circles.

% University Press Website:
% - Use CMYK (can be verified using Scribus > Edit > Colors
% - Sum of components should not exceed 300%
% A beautiful green color scheme

% Meant for top wave (the upper part of the front cover).
% Component sum < 3? OK
\definecolor{ColorScheme2DarkGreen}{cmyk}{0.453,0,0.734,0.749}
% Meant for front and back cover background.
% Component sum < 3? OK 
\definecolor{ColorScheme2DarkerGreen}{cmyk}{0.519,0,0.846,0.796}
% Meant for binding background.
% Component sum < 3? OK
% Calculated from ColorScheme2DarkerGreen by lowering the lightness value in the HSL model.
\definecolor{ColorScheme2EvenDarkerGreen}{cmyk}{0.513,0,0.846,0.847}
% Meant for wave that is mostly covered and for the circles.
% Component sum < 3? OK
% Calculated from ColorScheme2DarkerGreen by lowering the lightness value in the HSL model.
\definecolor{ColorScheme2Gray}{cmyk}{0.194,0.129,0,0.455}

% Define a color scheme above and then fill in here.
\definecolor{MyBindingBackgroundColor}{named}{ColorScheme2EvenDarkerGreen}
\definecolor{MyFrontCoverBackgroundColor}{named}{ColorScheme2DarkerGreen}
\definecolor{MyBackCoverBackgroundColor}{named}{ColorScheme2DarkerGreen}
\definecolor{FrontCoverWave1Color}{named}{ColorScheme2Gray}
\definecolor{FrontCoverWave2Color}{named}{ColorScheme2DarkGreen}

\definecolor{MyBindingTextColor}{named}{white}
\definecolor{MyFrontCoverTextColor}{named}{white}


% Best not to make the circles too distracting.
% Better pick a color that is used elsewhere too.
\definecolor{MyCirclesColor}{named}{FrontCoverWave1Color}
% And use some transparancy.
\newcommand{\MyCirclesOpacity}{0.5}

% Commands for fonts
% The \selectfont stuff
\newcommand{\MyFrontCoverFont}{\fcofamily}
\newcommand{\MyBindingFont}{\fcofamily}

\pagestyle{empty}

\begin{document}

\begin{tikzpicture}[remember picture,overlay,shift={(current page.south west)}]
% Set the bottom left of the pdf paper to be the origin (0,0) using shift.
% Define the coordinates marked in the picture
\coordinate (cPrime)                at (\MyBleedWidth,0);
\coordinate (dPrime)                at (\MyPdfPaperWidth - \MyBleedWidth,0);

\coordinate (iPrime)                at ([xshift=\MyBookPaperWidth]cPrime);
\coordinate (lPrime)                at ([xshift=-\MyBookPaperWidth]dPrime);

\coordinate (jPrime)                at ([xshift=\MyBindingCorrectionWidth]iPrime);
\coordinate (kPrime)                at ([xshift=-\MyBindingCorrectionWidth]lPrime);

\coordinate (aPrime)                at ([yshift=\MyPdfPaperHeight]cPrime);
\coordinate (bPrime)                at ([yshift=\MyPdfPaperHeight]dPrime);

\coordinate (ePrime)                at ([yshift=\MyPdfPaperHeight]iPrime);
\coordinate (hPrime)                at ([yshift=\MyPdfPaperHeight]lPrime);

\coordinate (fPrime)                at ([yshift=\MyPdfPaperHeight]jPrime);
\coordinate (gPrime)                at ([yshift=\MyPdfPaperHeight]kPrime);

\coordinate (BackCoverBottomLeft)   at ([yshift=\MyBleedHeight]cPrime); % (c)
\coordinate (FrontCoverBottomRight) at ([yshift=\MyBleedHeight]dPrime); % (d)

\coordinate (BackCoverBottomRight)  at ([yshift=\MyBleedHeight]iPrime); % (i)
\coordinate (FrontCoverBottomLeft)  at ([yshift=\MyBleedHeight]lPrime); % (l)

\coordinate (BindingBottomLeft)     at ([yshift=\MyBleedHeight]jPrime); % (j)
\coordinate (BindingBottomRight)    at ([yshift=\MyBleedHeight]kPrime); % (k)

\coordinate (BackCoverTopLeft)      at ([yshift=\MyBookPaperHeight]BackCoverBottomLeft); % (a)
\coordinate (FrontCoverTopRight)    at ([yshift=\MyBookPaperHeight]FrontCoverBottomRight); % (b)

\coordinate (BackCoverTopRight)     at ([yshift=\MyBookPaperHeight]BackCoverBottomRight); % (e)
\coordinate (FrontCoverTopLeft)     at ([yshift=\MyBookPaperHeight]FrontCoverBottomLeft); % (h)

\coordinate (BindingTopLeft)        at ([yshift=\MyBookPaperHeight]BindingBottomLeft); % (f)
\coordinate (BindingTopRight)       at ([yshift=\MyBookPaperHeight]BindingBottomRight); % (g)

% Coordinates not marked in the picture, but still necessary.
\coordinate (pdfBottomLeft)  at (0,0);
\coordinate (pdfTopLeft)     at ([yshift=\MyPdfPaperHeight]pdfBottomLeft);
\coordinate (pdfBottomRight) at ([xshift=\MyPdfPaperWidth]pdfBottomLeft);
\coordinate (pdfTopRight)    at ([yshift=\MyPdfPaperHeight]pdfBottomRight);

% Define binding diagonals. This does not draw anything.
% We use the newcommand so that we can construct the same path again without duplication.
% There are some tricks on Stackoverflow to save the path, but I don't understand them so I don't want to use them. I tried some though to no good effect.
\newcommand{\BindingMajorDiagonalDef}{(BindingBottomRight) -- (BindingTopLeft)}
\newcommand{\BindingMinorDiagonalDef}{(BindingBottomLeft) -- (BindingTopRight)}
\path[name path=BindingMajorDiagonal]\BindingMajorDiagonalDef;
\path[name path=BindingMinorDiagonal]\BindingMinorDiagonalDef;

% Define front cover diagonals. This does not draw anything.
\newcommand{\FrontCoverMajorDiagonalDef}{(FrontCoverBottomRight) -- (FrontCoverTopLeft)}
\newcommand{\FrontCoverMinorDiagonalDef}{(FrontCoverBottomLeft) -- (FrontCoverTopRight)}
\path[name path=FrontCoverMajorDiagonal]\FrontCoverMajorDiagonalDef;
\path[name path=FrontCoverMinorDiagonal]\FrontCoverMinorDiagonalDef;

% Center of the binding: intersection of the two diagonals
% In general two paths can have more than one intersection, so there probably is no shorter way.
% See TikZ manual page 57.
\path[name intersections={of=BindingMajorDiagonal and BindingMinorDiagonal}];
\coordinate (BindingCenter) at (intersection-1);
% \coordinate (BindingCenter) at (intersection of BindingBottomRight--BindingTopLeft and BindingBottomLeft--BindingTopRight);

% Center of the front cover: intersection of the two diagonals
\path[name intersections={of=FrontCoverMajorDiagonal and FrontCoverMinorDiagonal}];
\coordinate (FrontCoverCenter) at (intersection-1);

%%%%%%%%%%%%%%%%%%%%%%%%%%%%%%%%%%%%%%%%%%%%%%%%%%%%%%%%%%%%%%%%%%%%%%%%%%%%%%%%
% Draw auxiliary lines and points: COMMENT OUT IN FINAL VERSION
%%%%%%%%%%%%%%%%%%%%%%%%%%%%%%%%%%%%%%%%%%%%%%%%%%%%%%%%%%%%%%%%%%%%%%%%%%%%%%%%
%\begin{pgfonlayer}{auxiliaryLines}
%% Draw a large circle on the origin.
%%\draw (0,0) circle[radius=3pt];
%% Draw the coordinates to make sure they are in the correct place.
%\draw (cPrime) circle[radius=1pt];
%\draw (dPrime) circle[radius=1pt];
%\draw (iPrime) circle[radius=1pt];
%\draw (lPrime) circle[radius=1pt];
%\draw (jPrime) circle[radius=1pt];
%\draw (kPrime) circle[radius=1pt];
%
%\draw (aPrime) circle[radius=1pt];
%\draw (bPrime) circle[radius=1pt];
%\draw (ePrime) circle[radius=1pt];
%\draw (hPrime) circle[radius=1pt];
%\draw (fPrime) circle[radius=1pt];
%\draw (gPrime) circle[radius=1pt];
%
%\draw (BackCoverBottomLeft) circle[radius=1pt];
%\draw (FrontCoverBottomRight) circle[radius=1pt];
%\draw (BackCoverTopLeft) circle[radius=1pt];
%\draw (FrontCoverTopRight) circle[radius=1pt];
%\draw (BackCoverBottomRight) circle[radius=1pt];
%\draw (FrontCoverBottomLeft) circle[radius=1pt];
%\draw (BackCoverTopRight) circle[radius=1pt];
%\draw (FrontCoverTopLeft) circle[radius=1pt];
%\draw (BindingBottomLeft) circle[radius=1pt];
%\draw (BindingBottomRight) circle[radius=1pt];
%\draw (BindingTopLeft) circle[radius=1pt];
%\draw (BindingTopRight) circle[radius=1pt];
%
%% Binding diagonals and center point
%\draw \BindingMajorDiagonalDef;
%\draw \BindingMinorDiagonalDef;
%\draw (BindingCenter) circle[radius=4pt];
%
%% Front cover diagonals and center point
%\draw \FrontCoverMajorDiagonalDef;
%\draw \FrontCoverMinorDiagonalDef;
%\draw (FrontCoverCenter) circle[radius=4pt];
%
%% Box around the cover page to see the position of the text.
%\draw (FrontCoverTopLeft) -- (FrontCoverTopRight) --
%      (FrontCoverBottomRight) -- (FrontCoverBottomLeft) -- cycle;
%
%% Box around the binding to see the position of the text.
%\draw (BindingTopLeft) -- (BindingTopRight) --
%      (BindingBottomRight) -- (BindingBottomLeft) -- cycle;
%
%\end{pgfonlayer}
%%%%%%%%%%%%%%%%%%%%%%%%%%%%%%%%%%%%%%%%%%%%%%%%%%%%%%%%%%%%%%%%%%%%%%%%%%%%%%%%
% END auxiliary lines
%%%%%%%%%%%%%%%%%%%%%%%%%%%%%%%%%%%%%%%%%%%%%%%%%%%%%%%%%%%%%%%%%%%%%%%%%%%%%%%%
\begin{pgfonlayer}{background}
%%%%%%%%%%%%%%%%%%%%%%%%%%%%%%%%%%%%%%%%%%%%%%%%%%%%%%%%%%%%%%%%%%%%%%%%%%%%%%%%
% Binding background
%%%%%%%%%%%%%%%%%%%%%%%%%%%%%%%%%%%%%%%%%%%%%%%%%%%%%%%%%%%%%%%%%%%%%%%%%%%%%%%%
\fill[MyBindingBackgroundColor]
     (ePrime) -- (hPrime) -- (lPrime) -- (iPrime) -- cycle;
%%%%%%%%%%%%%%%%%%%%%%%%%%%%%%%%%%%%%%%%%%%%%%%%%%%%%%%%%%%%%%%%%%%%%%%%%%%%%%%%
% END Binding background
%%%%%%%%%%%%%%%%%%%%%%%%%%%%%%%%%%%%%%%%%%%%%%%%%%%%%%%%%%%%%%%%%%%%%%%%%%%%%%%%
%%%%%%%%%%%%%%%%%%%%%%%%%%%%%%%%%%%%%%%%%%%%%%%%%%%%%%%%%%%%%%%%%%%%%%%%%%%%%%%%
% Front cover background
%%%%%%%%%%%%%%%%%%%%%%%%%%%%%%%%%%%%%%%%%%%%%%%%%%%%%%%%%%%%%%%%%%%%%%%%%%%%%%%%
% Background color of the front page, visible at the bottom.
\fill[MyFrontCoverBackgroundColor]
     (hPrime) -- (pdfTopRight) -- (pdfBottomRight) -- (lPrime) -- cycle;
% Coordinates for the waves at the top of the front page. Random for now.
\coordinate (FrontCoverWave1Left) at ([yshift=-10cm]FrontCoverTopLeft);
\coordinate (FrontCoverWave1Right) at ([yshift=-8cm]pdfTopRight);
\coordinate (FrontCoverWave2Left) at (FrontCoverWave1Left);
\coordinate (FrontCoverWave2Right) at ([yshift=-7cm]pdfTopRight);
% Waves at the top of the front page.
% Based on http://tex.stackexchange.com/questions/175916/cover-page-with-latex
% The most interesting here are the angles.
\fill[FrontCoverWave1Color]
     (hPrime) -- (pdfTopRight) --
     (FrontCoverWave1Right) to[out=195,in=-15] (FrontCoverWave1Left);
\fill[FrontCoverWave2Color]
     (hPrime) -- (pdfTopRight) --
     (FrontCoverWave2Right) to[out=195,in=-15] (FrontCoverWave2Left);
%%%%%%%%%%%%%%%%%%%%%%%%%%%%%%%%%%%%%%%%%%%%%%%%%%%%%%%%%%%%%%%%%%%%%%%%%%%%%%%%
% END Front cover background
%%%%%%%%%%%%%%%%%%%%%%%%%%%%%%%%%%%%%%%%%%%%%%%%%%%%%%%%%%%%%%%%%%%%%%%%%%%%%%%%
%%%%%%%%%%%%%%%%%%%%%%%%%%%%%%%%%%%%%%%%%%%%%%%%%%%%%%%%%%%%%%%%%%%%%%%%%%%%%%%%
% Back cover background
%%%%%%%%%%%%%%%%%%%%%%%%%%%%%%%%%%%%%%%%%%%%%%%%%%%%%%%%%%%%%%%%%%%%%%%%%%%%%%%%
\fill[MyBackCoverBackgroundColor]
     (pdfTopLeft) -- (ePrime) -- (iPrime) -- (pdfBottomLeft) -- cycle;
%%%%%%%%%%%%%%%%%%%%%%%%%%%%%%%%%%%%%%%%%%%%%%%%%%%%%%%%%%%%%%%%%%%%%%%%%%%%%%%%
% END Back cover background
%%%%%%%%%%%%%%%%%%%%%%%%%%%%%%%%%%%%%%%%%%%%%%%%%%%%%%%%%%%%%%%%%%%%%%%%%%%%%%%%
\end{pgfonlayer}
\begin{pgfonlayer}{circles}
%%%%%%%%%%%%%%%%%%%%%%%%%%%%%%%%%%%%%%%%%%%%%%%%%%%%%%%%%%%%%%%%%%%%%%%%%%%%%%%%
% Front cover circles
%%%%%%%%%%%%%%%%%%%%%%%%%%%%%%%%%%%%%%%%%%%%%%%%%%%%%%%%%%%%%%%%%%%%%%%%%%%%%%%%
% The origin of the polar coordinate system that I will use for the circle animation.
\coordinate (FrontCoverCirclesOrigin) at (FrontCoverBottomRight);
\newcommand{\drawCircles}[1]{
  % Scope in which I'll install a temporary polar coordinate system with the origin at (FrontCoverCirclesOrigin) in the original coordinate system.
  \begin{scope}[shift=(#1)]
  % Small circles -- draw the full circle so that we are robust against slightly wider bleed. You'll never see the entire circle with realistic bleed widths.
  \foreach \i in {0,30,...,360} {
    \fill[fill=MyCirclesColor,opacity=\MyCirclesOpacity] (\i:1cm) circle[radius=5pt];
  }
  % Medium circles, note that they won't be on a line with the small ones due to 5 degree shift.
  % Medium circles on purpose do not have size 7.5.
  \foreach \i in {-5,10,...,355} {
    \fill[fill=MyCirclesColor,opacity=\MyCirclesOpacity] (\i:2.5cm) circle[radius=7pt];
  }
  % Large circles, note that these too won't be on a line with the medium ones.
  \foreach \i in {-10,5,...,350} {
    \fill[fill=MyCirclesColor,opacity=\MyCirclesOpacity] (\i:5cm) circle[radius=10pt];
  }
  \end{scope}
} % END newcommand
\drawCircles{FrontCoverCirclesOrigin}
%%%%%%%%%%%%%%%%%%%%%%%%%%%%%%%%%%%%%%%%%%%%%%%%%%%%%%%%%%%%%%%%%%%%%%%%%%%%%%%%
% END Front cover circles
%%%%%%%%%%%%%%%%%%%%%%%%%%%%%%%%%%%%%%%%%%%%%%%%%%%%%%%%%%%%%%%%%%%%%%%%%%%%%%%%
%%%%%%%%%%%%%%%%%%%%%%%%%%%%%%%%%%%%%%%%%%%%%%%%%%%%%%%%%%%%%%%%%%%%%%%%%%%%%%%%
% Back cover circles
%%%%%%%%%%%%%%%%%%%%%%%%%%%%%%%%%%%%%%%%%%%%%%%%%%%%%%%%%%%%%%%%%%%%%%%%%%%%%%%%
\coordinate (BackCoverCirclesOrigin) at (BackCoverTopLeft);
% Flip the circles drawn on the front page over both axes by inverting the axes.
\begin{scope}[xscale=-1,yscale=-1]
\drawCircles{BackCoverCirclesOrigin}
\end{scope}
%%%%%%%%%%%%%%%%%%%%%%%%%%%%%%%%%%%%%%%%%%%%%%%%%%%%%%%%%%%%%%%%%%%%%%%%%%%%%%%%
% END Back cover circles
%%%%%%%%%%%%%%%%%%%%%%%%%%%%%%%%%%%%%%%%%%%%%%%%%%%%%%%%%%%%%%%%%%%%%%%%%%%%%%%%
\end{pgfonlayer}
\begin{pgfonlayer}{foreground}
%%%%%%%%%%%%%%%%%%%%%%%%%%%%%%%%%%%%%%%%%%%%%%%%%%%%%%%%%%%%%%%%%%%%%%%%%%%%%%%%
% Binding foreground
%%%%%%%%%%%%%%%%%%%%%%%%%%%%%%%%%%%%%%%%%%%%%%%%%%%%%%%%%%%%%%%%%%%%%%%%%%%%%%%%
% Text positions -- random for now
\coordinate (BindingTitlePosition) at ([yshift=2cm]BindingCenter);
\coordinate (BindingAuthorPosition) at ([yshift=-10cm]BindingCenter);

\node[color=MyBindingTextColor,rotate=270,font=\LARGE\MyBindingFont] at (BindingTitlePosition) {\MyBindingTitle};
\node[color=MyBindingTextColor,rotate=270,font=\MyBindingFont] at (BindingAuthorPosition) {\MyBindingName};
%%%%%%%%%%%%%%%%%%%%%%%%%%%%%%%%%%%%%%%%%%%%%%%%%%%%%%%%%%%%%%%%%%%%%%%%%%%%%%%%
% END Binding foreground
%%%%%%%%%%%%%%%%%%%%%%%%%%%%%%%%%%%%%%%%%%%%%%%%%%%%%%%%%%%%%%%%%%%%%%%%%%%%%%%%
%%%%%%%%%%%%%%%%%%%%%%%%%%%%%%%%%%%%%%%%%%%%%%%%%%%%%%%%%%%%%%%%%%%%%%%%%%%%%%%%
% Front cover foreground
%%%%%%%%%%%%%%%%%%%%%%%%%%%%%%%%%%%%%%%%%%%%%%%%%%%%%%%%%%%%%%%%%%%%%%%%%%%%%%%%
% Text positions -- random for now
\coordinate (FrontCoverTitlePosition) at (FrontCoverCenter);
\coordinate (FrontCoverAuthorPosition) at ([yshift=-5cm]FrontCoverCenter);
% Logo position -- random for now
\coordinate (FrontCoverLogoPosition) at ([yshift=5cm]FrontCoverCenter);
% The align key is the key to allowing linebreaks (\\) in the title.
\node at (FrontCoverLogoPosition) {\includegraphics{\LogoFile}};
\node[color=MyFrontCoverTextColor,align=center,font=\LARGE\MyFrontCoverFont] at (FrontCoverTitlePosition) {\MyFrontCoverTitle};
\node[color=MyFrontCoverTextColor,font=\LARGE\MyFrontCoverFont] at (FrontCoverAuthorPosition) {\MyFrontCoverName};
%%%%%%%%%%%%%%%%%%%%%%%%%%%%%%%%%%%%%%%%%%%%%%%%%%%%%%%%%%%%%%%%%%%%%%%%%%%%%%%%
% END Front cover foreground
%%%%%%%%%%%%%%%%%%%%%%%%%%%%%%%%%%%%%%%%%%%%%%%%%%%%%%%%%%%%%%%%%%%%%%%%%%%%%%%%
% \node[
%   align=left,
%   font=\color{white}\fontsize{58}{64}\bfseries,
%   anchor=north
%   ]
%   at ([yshift=-2.3cm]current page.north)
%   (name)
%   {\TeX \\[0.3ex] \LaTeX};
\end{pgfonlayer}
\end{tikzpicture}

\end{document}
