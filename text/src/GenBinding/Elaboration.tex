\chapter{Elaboration}\label{ch:elaboration}
For the meta-theoretical formalization of programming language we need
properties of the semantical definitions. For instance, for type-theory proofs
via progress and preservation we generally need substitution properties of the
typing relation, and all the other boilerplate that is necessary to prove it.

There are multiple ways to the boilerplate lemmas for use in mechanizations. We
can develop a \emph{code generator} that produces code for a proof-assistant.
Alternatively, we can develop a \emph{datatype-generic} library inside a
proof-assistant. Both approaches have different trade-offs.

%% \item Programming language mechanizations typically rely on many boilerplate
%%   properties of the boilerplate operations that we introduced in the previous
%%   chapter.

\paragraph{Generic Library}

Alternatively, we can with (well-formed) \Knot specifications as the codes of a
generic universe. The main benefit of a library is, that its code, including all
proofs, can be checked separately independently of a concrete specification.
Hence, instantiating the library is guaranteed to yield valid theorems for all
specifications.

Advantages:
\begin{itemize}
\item First, \Yarn provides evidence for the existence of type-sound boilerplate
  definitions and proofs for for every language specified with \Knot.
\item It shows coherence of the specification language itself, all information
  is contained in the universe codes of \Yarn, i.e. in \Knot specifications.
\end{itemize}

Disadvantages of a generic library:
\begin{itemize}
\item Hiding of the internals. Do not expose the user of the library
  to its complexity. Ideally, the user does not have to be an expert
  in datatype-generic programming to use the library.

\item Generically handling relations with arbitrary arities is challenging.

\item Induction over the generic generic representation is usually not the
  desired one. Do not use datatypes of the extension directly, but instead write
  isomorphism to user defined types. This also applies to well-scopedness
  predicates and relations.

\item Generic implementation shines through. For example, during
  simplification, the generic implementation might be exposed.

\item The statement of a generic lemma may not unify with all goals it applies
  to, before it has been properly specialized to a given specification. This
  impacts automatic proof search.

\item Related to that is the problem of proof discovery. Conceptually, both pose
  not a problem since in theory a proof assistant could specialize the types of
  lemmas when instantiating the generic code for a specification. However,
  currently none of the available proof assistants provide such functionality.
\end{itemize}


\paragraph{Code Generators}
A code generator can parse textual \Knot specifications and produces code for
use in a proof-assistant. In terms of usability, there are clear advantages.

The code generator can produce definition of datatypes and functions that are
close to what a human prover might write manually. Hence, the user can inspect
the code to understand the provided functionality in detail. Moreover, since the
code generator can specialize the statements of lemmas to the given
specification, it is much easier for the user to look at the provided proofs
and for proof automation to apply the provided proofs.

The main downside of a code generator is that the output still needs to be
checked. This raises the question if correct code is generated in all cases.
There are multiple ways to implement proofs; depending on the available methods
in the proof assistant. For instance, \Coq has a powerful scripting language to
automate writing proofs. However, very generic proof scripts tend to be brittle
and can possibly take a long time to execute. Furthermore, reasoning about the
correctness of the code generator not only involves the language of the proof
assistant, but also that of the scripting system.


\paragraph{Our Approach}

The main trade-off between the approaches is the confidence in a generic
implementation and the usability of a code generator. Ideally, we could combine
both approaches in order to get the advantages of either one, and also mitigate
their disadvantages.

This is the approach that we take. More specifically, we implement both a
generic implementation of \Knot in \Coq, called \Yarn, and a code generator
named \Needle that produces \Coq code and is implemented in Haskell. Even though
both implementations do not share code directly, we try to transfer as much
confidence as possible from \Yarn to \Needle. We outline our methodology to this
below. Alternatively, one could imagine a datatype-generic core library and a
code generator around it that takes care of the usability problem. We come back
to this idea in the discussion of this chapter in Section
\ref{sec:elab:discussion}.

\Yarn defines de Bruijn interpretation of \Knot specification generically and
also implemented boilerplate functions and lemmas generically. To derive a code
generator based on \Yarn's code, generic definitions needs to be expanded into
specialized code for a particular specification, or put differently, we need to
turn the generic implementation into an elaboration into the target language.
Elaborating datatype and function definitions is comparably easy. Much more
challenging is to elaborate a specification into boilerplate lemmas and their
proofs. This will be the focus of this chapter. The boilerplate lemmas that are
implemented in \Yarn and \Needle are not novel. The novelties lie in our
elaboration of them. Hence we specifically turn our attention to the elaboration
methodology.


\paragraph{Methodology}
The main obstacle, that we need to solve, is how to turn a generic proof of
\Yarn into an elaboration function in the code generator \Needle. One can aim to
implement an elaboration that follows the \emph{same strategy as the generic
  proof}. However, the question remains if this is a faithful implementation of
the same proof and if it indeed is correct in all cases, especially when proof
scripts are used.

We solve this problem by factoring the generic proofs through domain-specific
witness languages, i.e. rather than implementing the proofs directly, we
implement generic and formal elaborations into an intermediate language of
proofs.

This splits the generic proof into multiple distinct parts:
\begin{enumerate}
\item A (syntax-directed) elaboration into a witness language that we designed
  for the given class of lemmas.
\item A semantics of the witness langauge.
\item The correctness of the elaboration, i.e. the produced witness indeed
  encodes a proof of the desired lemma w.r.t. the chosen semantics.
\item A soundness proof of the semantics of the witness language, e.g. the
  statement that a witness proves w.r.t. to the chosen semantics is valid (in
  our meta-language).
\end{enumerate}

The benefit of this approach is that we can give a formal definition of
elaborations and analyze the elaborations formally, i.e. formally proof their
correctness. Subsequently, we can implement the same elaborations in the code
generator. The correctness of \Needle still relies on some factors

\begin{enumerate}
\item \emph{The elaboration functions from \Knot specifications to the witness
  languages is correctly ported from \Coq to \Haskell.} The elaborations are
  recursive and pure functions\footnote{In reality, the elaboration functions
    are exclusively folds.} over algebraic dataypes which both \Coq and \Haskell
  support. Therefore, we port the elaboration code by direct translation from
  \Coq to \Haskell. In fact, most \Haskell elaboration functions were derived
  by a copy \& paste of the \Coq code.
\item \emph{The elaboration of the witness language into the language of the
  target proof-assistant is correctly implemented.} For our witness languages,
  this translation is straightforward, since each witness primitive is either
  implemented by an application of a lemma or a rule.
\item \emph{The impact of any differences, between \Needle and \Yarn.}
  \stevennote{TODO}{Discuss differences between \Needle and \Yarn. Glue code not
    verified.}
\end{enumerate}



\paragraph{Overview}

There are too many lemmas to give an exhaustive account of all the
necessary elaborations. Instead we concentrate on representative and key lemmas.

The sections in this chapter show how the witness language methodology is
applied to several classes of boilerplate lemmas. Section
\ref{ssec:elab:interaction:overview} discusses common interaction lemmas between
syntactic operations. Section \ref{sec:elab:wellscopedness} deals with the
well-scopedness lemmas that we discussed in Section
\ref{sec:gen:overview:formalization:semantics} and
\ref{sec:gen:formalization:metatheory}, i.e. lemmas that prove that indices of
relations are always well-scoped. Section \ref{sec:elab:shifting} covers
shifting and substitution lemmas for relations. Section
\ref{sec:elab:impl:generic} discusses a generic implementation of \Knot in \Coq
and generic proofs of lemmas that follow the methodology of this chapter.
Finally, Section \ref{sec:elab:impl:codegen} covers the implementation of our
code generator \Needle.


%% \stevennote{MOVE}{To further reduce the hand-written boilerplate, we have set up
%%   the \Knot specification language in such a way that it provides all the
%%   necessary information to generically state and prove a wide range of these
%%   properties.}

%% \footnote{In fact, we provide more such lemmas than any other framework based on
%%   first-order representations -- see Section~\ref{sec:related}.}

%% \stevennote{MOVE}{Below we discuss different kinds of ubiquitous lemmas that
%%   \Needle covers; Appendix~\ref{app:lemmas} provides more detail.}

%% \stevennote{MOVE}{It is quite challenging to tackle these boilerplate lemmas
%%   generically because their exact statements, and in particular which premises
%%   are needed, depend highly on the dependencies between sorts and of the
%%   associated data in environments.}

\input{src/GenBinding/Elaboration/Interaction} \clearpage
\input{src/GenBinding/Elaboration/WellScopedness}
\input{src/GenBinding/Elaboration/RelShifting}
\input{src/GenBinding/Elaboration/RelSubstitution}
\input{src/GenBinding/Elaboration/Implementation}
%% ODER: format ==         = "\mathrel{==}"
%% ODER: format /=         = "\neq "
%
%
\makeatletter
\@ifundefined{lhs2tex.lhs2tex.sty.read}%
  {\@namedef{lhs2tex.lhs2tex.sty.read}{}%
   \newcommand\SkipToFmtEnd{}%
   \newcommand\EndFmtInput{}%
   \long\def\SkipToFmtEnd#1\EndFmtInput{}%
  }\SkipToFmtEnd

\newcommand\ReadOnlyOnce[1]{\@ifundefined{#1}{\@namedef{#1}{}}\SkipToFmtEnd}
\usepackage{amstext}
\usepackage{amssymb}
\usepackage{stmaryrd}
\DeclareFontFamily{OT1}{cmtex}{}
\DeclareFontShape{OT1}{cmtex}{m}{n}
  {<5><6><7><8>cmtex8
   <9>cmtex9
   <10><10.95><12><14.4><17.28><20.74><24.88>cmtex10}{}
\DeclareFontShape{OT1}{cmtex}{m}{it}
  {<-> ssub * cmtt/m/it}{}
\newcommand{\texfamily}{\fontfamily{cmtex}\selectfont}
\DeclareFontShape{OT1}{cmtt}{bx}{n}
  {<5><6><7><8>cmtt8
   <9>cmbtt9
   <10><10.95><12><14.4><17.28><20.74><24.88>cmbtt10}{}
\DeclareFontShape{OT1}{cmtex}{bx}{n}
  {<-> ssub * cmtt/bx/n}{}
\newcommand{\tex}[1]{\text{\texfamily#1}}	% NEU

\newcommand{\Sp}{\hskip.33334em\relax}


\newcommand{\Conid}[1]{\mathit{#1}}
\newcommand{\Varid}[1]{\mathit{#1}}
\newcommand{\anonymous}{\kern0.06em \vbox{\hrule\@width.5em}}
\newcommand{\plus}{\mathbin{+\!\!\!+}}
\newcommand{\bind}{\mathbin{>\!\!\!>\mkern-6.7mu=}}
\newcommand{\rbind}{\mathbin{=\mkern-6.7mu<\!\!\!<}}% suggested by Neil Mitchell
\newcommand{\sequ}{\mathbin{>\!\!\!>}}
\renewcommand{\leq}{\leqslant}
\renewcommand{\geq}{\geqslant}
\usepackage{polytable}

%mathindent has to be defined
\@ifundefined{mathindent}%
  {\newdimen\mathindent\mathindent\leftmargini}%
  {}%

\def\resethooks{%
  \global\let\SaveRestoreHook\empty
  \global\let\ColumnHook\empty}
\newcommand*{\savecolumns}[1][default]%
  {\g@addto@macro\SaveRestoreHook{\savecolumns[#1]}}
\newcommand*{\restorecolumns}[1][default]%
  {\g@addto@macro\SaveRestoreHook{\restorecolumns[#1]}}
\newcommand*{\aligncolumn}[2]%
  {\g@addto@macro\ColumnHook{\column{#1}{#2}}}

\resethooks

\newcommand{\onelinecommentchars}{\quad-{}- }
\newcommand{\commentbeginchars}{\enskip\{-}
\newcommand{\commentendchars}{-\}\enskip}

\newcommand{\visiblecomments}{%
  \let\onelinecomment=\onelinecommentchars
  \let\commentbegin=\commentbeginchars
  \let\commentend=\commentendchars}

\newcommand{\invisiblecomments}{%
  \let\onelinecomment=\empty
  \let\commentbegin=\empty
  \let\commentend=\empty}

\visiblecomments

\newlength{\blanklineskip}
\setlength{\blanklineskip}{0.66084ex}

\newcommand{\hsindent}[1]{\quad}% default is fixed indentation
\let\hspre\empty
\let\hspost\empty
\newcommand{\NB}{\textbf{NB}}
\newcommand{\Todo}[1]{$\langle$\textbf{To do:}~#1$\rangle$}

\EndFmtInput
\makeatother
%
%
%
%
%
%
% This package provides two environments suitable to take the place
% of hscode, called "plainhscode" and "arrayhscode". 
%
% The plain environment surrounds each code block by vertical space,
% and it uses \abovedisplayskip and \belowdisplayskip to get spacing
% similar to formulas. Note that if these dimensions are changed,
% the spacing around displayed math formulas changes as well.
% All code is indented using \leftskip.
%
% Changed 19.08.2004 to reflect changes in colorcode. Should work with
% CodeGroup.sty.
%
\ReadOnlyOnce{polycode.fmt}%
\makeatletter

\newcommand{\hsnewpar}[1]%
  {{\parskip=0pt\parindent=0pt\par\vskip #1\noindent}}

% can be used, for instance, to redefine the code size, by setting the
% command to \small or something alike
\newcommand{\hscodestyle}{}

% The command \sethscode can be used to switch the code formatting
% behaviour by mapping the hscode environment in the subst directive
% to a new LaTeX environment.

\newcommand{\sethscode}[1]%
  {\expandafter\let\expandafter\hscode\csname #1\endcsname
   \expandafter\let\expandafter\endhscode\csname end#1\endcsname}

% "compatibility" mode restores the non-polycode.fmt layout.

\newenvironment{compathscode}%
  {\par\noindent
   \advance\leftskip\mathindent
   \hscodestyle
   \let\\=\@normalcr
   \let\hspre\(\let\hspost\)%
   \pboxed}%
  {\endpboxed\)%
   \par\noindent
   \ignorespacesafterend}

\newcommand{\compaths}{\sethscode{compathscode}}

% "plain" mode is the proposed default.
% It should now work with \centering.
% This required some changes. The old version
% is still available for reference as oldplainhscode.

\newenvironment{plainhscode}%
  {\hsnewpar\abovedisplayskip
   \advance\leftskip\mathindent
   \hscodestyle
   \let\hspre\(\let\hspost\)%
   \pboxed}%
  {\endpboxed%
   \hsnewpar\belowdisplayskip
   \ignorespacesafterend}

\newenvironment{oldplainhscode}%
  {\hsnewpar\abovedisplayskip
   \advance\leftskip\mathindent
   \hscodestyle
   \let\\=\@normalcr
   \(\pboxed}%
  {\endpboxed\)%
   \hsnewpar\belowdisplayskip
   \ignorespacesafterend}

% Here, we make plainhscode the default environment.

\newcommand{\plainhs}{\sethscode{plainhscode}}
\newcommand{\oldplainhs}{\sethscode{oldplainhscode}}
\plainhs

% The arrayhscode is like plain, but makes use of polytable's
% parray environment which disallows page breaks in code blocks.

\newenvironment{arrayhscode}%
  {\hsnewpar\abovedisplayskip
   \advance\leftskip\mathindent
   \hscodestyle
   \let\\=\@normalcr
   \(\parray}%
  {\endparray\)%
   \hsnewpar\belowdisplayskip
   \ignorespacesafterend}

\newcommand{\arrayhs}{\sethscode{arrayhscode}}

% The mathhscode environment also makes use of polytable's parray 
% environment. It is supposed to be used only inside math mode 
% (I used it to typeset the type rules in my thesis).

\newenvironment{mathhscode}%
  {\parray}{\endparray}

\newcommand{\mathhs}{\sethscode{mathhscode}}

% texths is similar to mathhs, but works in text mode.

\newenvironment{texthscode}%
  {\(\parray}{\endparray\)}

\newcommand{\texths}{\sethscode{texthscode}}

% The framed environment places code in a framed box.

\def\codeframewidth{\arrayrulewidth}
\RequirePackage{calc}

\newenvironment{framedhscode}%
  {\parskip=\abovedisplayskip\par\noindent
   \hscodestyle
   \arrayrulewidth=\codeframewidth
   \tabular{@{}|p{\linewidth-2\arraycolsep-2\arrayrulewidth-2pt}|@{}}%
   \hline\framedhslinecorrect\\{-1.5ex}%
   \let\endoflinesave=\\
   \let\\=\@normalcr
   \(\pboxed}%
  {\endpboxed\)%
   \framedhslinecorrect\endoflinesave{.5ex}\hline
   \endtabular
   \parskip=\belowdisplayskip\par\noindent
   \ignorespacesafterend}

\newcommand{\framedhslinecorrect}[2]%
  {#1[#2]}

\newcommand{\framedhs}{\sethscode{framedhscode}}

% The inlinehscode environment is an experimental environment
% that can be used to typeset displayed code inline.

\newenvironment{inlinehscode}%
  {\(\def\column##1##2{}%
   \let\>\undefined\let\<\undefined\let\\\undefined
   \newcommand\>[1][]{}\newcommand\<[1][]{}\newcommand\\[1][]{}%
   \def\fromto##1##2##3{##3}%
   \def\nextline{}}{\) }%

\newcommand{\inlinehs}{\sethscode{inlinehscode}}

% The joincode environment is a separate environment that
% can be used to surround and thereby connect multiple code
% blocks.

\newenvironment{joincode}%
  {\let\orighscode=\hscode
   \let\origendhscode=\endhscode
   \def\endhscode{\def\hscode{\endgroup\def\@currenvir{hscode}\\}\begingroup}
   %\let\SaveRestoreHook=\empty
   %\let\ColumnHook=\empty
   %\let\resethooks=\empty
   \orighscode\def\hscode{\endgroup\def\@currenvir{hscode}}}%
  {\origendhscode
   \global\let\hscode=\orighscode
   \global\let\endhscode=\origendhscode}%

\makeatother
\EndFmtInput
%

% format runC         = "run\mathbb{C}_{\scalebox{0.6}{T}}"
% format C            = "\mathbb{C}_{\scalebox{0.6}{T}}"
%%format R            = "\mathbb{R}_{\scalebox{0.6}{T}}"
















%% %format (evalsym bs sym th) = "{\evalsym{" bs "}{" sym "}{" th "}}"
%% %format (evalrbs (rbs) l e th) = "{\evalrbs{" rbs "}{" l "}{" e "}{" th "}}"

%% %format (evalsym (bs) sym th) = "\llbracket\: {" bs "} \mid {" sym "} \:\rrbracket_{" th "}"





%
%
% First, let's redefine the forall, and the dot.
%
%
% This is made in such a way that after a forall, the next
% dot will be printed as a period, otherwise the formatting
% of `comp_` is used. By redefining `comp_`, as suitable
% composition operator can be chosen. Similarly, period_
% is used for the period.
%
\ReadOnlyOnce{forall.fmt}%
\makeatletter

% The HaskellResetHook is a list to which things can
% be added that reset the Haskell state to the beginning.
% This is to recover from states where the hacked intelligence
% is not sufficient.

\let\HaskellResetHook\empty
\newcommand*{\AtHaskellReset}[1]{%
  \g@addto@macro\HaskellResetHook{#1}}
\newcommand*{\HaskellReset}{\HaskellResetHook}

\global\let\hsforallread\empty

\newcommand\hsforall{\global\let\hsdot=\hsperiodonce}
\newcommand*\hsperiodonce[2]{#2\global\let\hsdot=\hscompose}
\newcommand*\hscompose[2]{#1}

\AtHaskellReset{\global\let\hsdot=\hscompose}

% In the beginning, we should reset Haskell once.
\HaskellReset

\makeatother
\EndFmtInput

\section{Discussion}\label{sec:elab:discussion}

%%% Local Variables:
%%% mode: latex
%%% TeX-master: "../Main"
%%% End:


%%% Local Variables:
%%% mode: latex
%%% TeX-master: "../main"
%%% End:
