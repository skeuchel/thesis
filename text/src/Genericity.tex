
\chapter{Background}\label{ch:gen:background}
%% %% ODER: format ==         = "\mathrel{==}"
%% ODER: format /=         = "\neq "
%
%
\makeatletter
\@ifundefined{lhs2tex.lhs2tex.sty.read}%
  {\@namedef{lhs2tex.lhs2tex.sty.read}{}%
   \newcommand\SkipToFmtEnd{}%
   \newcommand\EndFmtInput{}%
   \long\def\SkipToFmtEnd#1\EndFmtInput{}%
  }\SkipToFmtEnd

\newcommand\ReadOnlyOnce[1]{\@ifundefined{#1}{\@namedef{#1}{}}\SkipToFmtEnd}
\usepackage{amstext}
\usepackage{amssymb}
\usepackage{stmaryrd}
\DeclareFontFamily{OT1}{cmtex}{}
\DeclareFontShape{OT1}{cmtex}{m}{n}
  {<5><6><7><8>cmtex8
   <9>cmtex9
   <10><10.95><12><14.4><17.28><20.74><24.88>cmtex10}{}
\DeclareFontShape{OT1}{cmtex}{m}{it}
  {<-> ssub * cmtt/m/it}{}
\newcommand{\texfamily}{\fontfamily{cmtex}\selectfont}
\DeclareFontShape{OT1}{cmtt}{bx}{n}
  {<5><6><7><8>cmtt8
   <9>cmbtt9
   <10><10.95><12><14.4><17.28><20.74><24.88>cmbtt10}{}
\DeclareFontShape{OT1}{cmtex}{bx}{n}
  {<-> ssub * cmtt/bx/n}{}
\newcommand{\tex}[1]{\text{\texfamily#1}}	% NEU

\newcommand{\Sp}{\hskip.33334em\relax}


\newcommand{\Conid}[1]{\mathit{#1}}
\newcommand{\Varid}[1]{\mathit{#1}}
\newcommand{\anonymous}{\kern0.06em \vbox{\hrule\@width.5em}}
\newcommand{\plus}{\mathbin{+\!\!\!+}}
\newcommand{\bind}{\mathbin{>\!\!\!>\mkern-6.7mu=}}
\newcommand{\rbind}{\mathbin{=\mkern-6.7mu<\!\!\!<}}% suggested by Neil Mitchell
\newcommand{\sequ}{\mathbin{>\!\!\!>}}
\renewcommand{\leq}{\leqslant}
\renewcommand{\geq}{\geqslant}
\usepackage{polytable}

%mathindent has to be defined
\@ifundefined{mathindent}%
  {\newdimen\mathindent\mathindent\leftmargini}%
  {}%

\def\resethooks{%
  \global\let\SaveRestoreHook\empty
  \global\let\ColumnHook\empty}
\newcommand*{\savecolumns}[1][default]%
  {\g@addto@macro\SaveRestoreHook{\savecolumns[#1]}}
\newcommand*{\restorecolumns}[1][default]%
  {\g@addto@macro\SaveRestoreHook{\restorecolumns[#1]}}
\newcommand*{\aligncolumn}[2]%
  {\g@addto@macro\ColumnHook{\column{#1}{#2}}}

\resethooks

\newcommand{\onelinecommentchars}{\quad-{}- }
\newcommand{\commentbeginchars}{\enskip\{-}
\newcommand{\commentendchars}{-\}\enskip}

\newcommand{\visiblecomments}{%
  \let\onelinecomment=\onelinecommentchars
  \let\commentbegin=\commentbeginchars
  \let\commentend=\commentendchars}

\newcommand{\invisiblecomments}{%
  \let\onelinecomment=\empty
  \let\commentbegin=\empty
  \let\commentend=\empty}

\visiblecomments

\newlength{\blanklineskip}
\setlength{\blanklineskip}{0.66084ex}

\newcommand{\hsindent}[1]{\quad}% default is fixed indentation
\let\hspre\empty
\let\hspost\empty
\newcommand{\NB}{\textbf{NB}}
\newcommand{\Todo}[1]{$\langle$\textbf{To do:}~#1$\rangle$}

\EndFmtInput
\makeatother
%
%
%
% First, let's redefine the forall, and the dot.
%
%
% This is made in such a way that after a forall, the next
% dot will be printed as a period, otherwise the formatting
% of `comp_` is used. By redefining `comp_`, as suitable
% composition operator can be chosen. Similarly, period_
% is used for the period.
%
\ReadOnlyOnce{forall.fmt}%
\makeatletter

% The HaskellResetHook is a list to which things can
% be added that reset the Haskell state to the beginning.
% This is to recover from states where the hacked intelligence
% is not sufficient.

\let\HaskellResetHook\empty
\newcommand*{\AtHaskellReset}[1]{%
  \g@addto@macro\HaskellResetHook{#1}}
\newcommand*{\HaskellReset}{\HaskellResetHook}

\global\let\hsforallread\empty

\newcommand\hsforall{\global\let\hsdot=\hsperiodonce}
\newcommand*\hsperiodonce[2]{#2\global\let\hsdot=\hscompose}
\newcommand*\hscompose[2]{#1}

\AtHaskellReset{\global\let\hsdot=\hscompose}

% In the beginning, we should reset Haskell once.
\HaskellReset

\makeatother
\EndFmtInput
%
%
%
%
%
% This package provides two environments suitable to take the place
% of hscode, called "plainhscode" and "arrayhscode". 
%
% The plain environment surrounds each code block by vertical space,
% and it uses \abovedisplayskip and \belowdisplayskip to get spacing
% similar to formulas. Note that if these dimensions are changed,
% the spacing around displayed math formulas changes as well.
% All code is indented using \leftskip.
%
% Changed 19.08.2004 to reflect changes in colorcode. Should work with
% CodeGroup.sty.
%
\ReadOnlyOnce{polycode.fmt}%
\makeatletter

\newcommand{\hsnewpar}[1]%
  {{\parskip=0pt\parindent=0pt\par\vskip #1\noindent}}

% can be used, for instance, to redefine the code size, by setting the
% command to \small or something alike
\newcommand{\hscodestyle}{}

% The command \sethscode can be used to switch the code formatting
% behaviour by mapping the hscode environment in the subst directive
% to a new LaTeX environment.

\newcommand{\sethscode}[1]%
  {\expandafter\let\expandafter\hscode\csname #1\endcsname
   \expandafter\let\expandafter\endhscode\csname end#1\endcsname}

% "compatibility" mode restores the non-polycode.fmt layout.

\newenvironment{compathscode}%
  {\par\noindent
   \advance\leftskip\mathindent
   \hscodestyle
   \let\\=\@normalcr
   \let\hspre\(\let\hspost\)%
   \pboxed}%
  {\endpboxed\)%
   \par\noindent
   \ignorespacesafterend}

\newcommand{\compaths}{\sethscode{compathscode}}

% "plain" mode is the proposed default.
% It should now work with \centering.
% This required some changes. The old version
% is still available for reference as oldplainhscode.

\newenvironment{plainhscode}%
  {\hsnewpar\abovedisplayskip
   \advance\leftskip\mathindent
   \hscodestyle
   \let\hspre\(\let\hspost\)%
   \pboxed}%
  {\endpboxed%
   \hsnewpar\belowdisplayskip
   \ignorespacesafterend}

\newenvironment{oldplainhscode}%
  {\hsnewpar\abovedisplayskip
   \advance\leftskip\mathindent
   \hscodestyle
   \let\\=\@normalcr
   \(\pboxed}%
  {\endpboxed\)%
   \hsnewpar\belowdisplayskip
   \ignorespacesafterend}

% Here, we make plainhscode the default environment.

\newcommand{\plainhs}{\sethscode{plainhscode}}
\newcommand{\oldplainhs}{\sethscode{oldplainhscode}}
\plainhs

% The arrayhscode is like plain, but makes use of polytable's
% parray environment which disallows page breaks in code blocks.

\newenvironment{arrayhscode}%
  {\hsnewpar\abovedisplayskip
   \advance\leftskip\mathindent
   \hscodestyle
   \let\\=\@normalcr
   \(\parray}%
  {\endparray\)%
   \hsnewpar\belowdisplayskip
   \ignorespacesafterend}

\newcommand{\arrayhs}{\sethscode{arrayhscode}}

% The mathhscode environment also makes use of polytable's parray 
% environment. It is supposed to be used only inside math mode 
% (I used it to typeset the type rules in my thesis).

\newenvironment{mathhscode}%
  {\parray}{\endparray}

\newcommand{\mathhs}{\sethscode{mathhscode}}

% texths is similar to mathhs, but works in text mode.

\newenvironment{texthscode}%
  {\(\parray}{\endparray\)}

\newcommand{\texths}{\sethscode{texthscode}}

% The framed environment places code in a framed box.

\def\codeframewidth{\arrayrulewidth}
\RequirePackage{calc}

\newenvironment{framedhscode}%
  {\parskip=\abovedisplayskip\par\noindent
   \hscodestyle
   \arrayrulewidth=\codeframewidth
   \tabular{@{}|p{\linewidth-2\arraycolsep-2\arrayrulewidth-2pt}|@{}}%
   \hline\framedhslinecorrect\\{-1.5ex}%
   \let\endoflinesave=\\
   \let\\=\@normalcr
   \(\pboxed}%
  {\endpboxed\)%
   \framedhslinecorrect\endoflinesave{.5ex}\hline
   \endtabular
   \parskip=\belowdisplayskip\par\noindent
   \ignorespacesafterend}

\newcommand{\framedhslinecorrect}[2]%
  {#1[#2]}

\newcommand{\framedhs}{\sethscode{framedhscode}}

% The inlinehscode environment is an experimental environment
% that can be used to typeset displayed code inline.

\newenvironment{inlinehscode}%
  {\(\def\column##1##2{}%
   \let\>\undefined\let\<\undefined\let\\\undefined
   \newcommand\>[1][]{}\newcommand\<[1][]{}\newcommand\\[1][]{}%
   \def\fromto##1##2##3{##3}%
   \def\nextline{}}{\) }%

\newcommand{\inlinehs}{\sethscode{inlinehscode}}

% The joincode environment is a separate environment that
% can be used to surround and thereby connect multiple code
% blocks.

\newenvironment{joincode}%
  {\let\orighscode=\hscode
   \let\origendhscode=\endhscode
   \def\endhscode{\def\hscode{\endgroup\def\@currenvir{hscode}\\}\begingroup}
   %\let\SaveRestoreHook=\empty
   %\let\ColumnHook=\empty
   %\let\resethooks=\empty
   \orighscode\def\hscode{\endgroup\def\@currenvir{hscode}}}%
  {\origendhscode
   \global\let\hscode=\orighscode
   \global\let\endhscode=\origendhscode}%

\makeatother
\EndFmtInput
%

\section{Introduction}
\label{sec:intro}

Theorem provers are actively used to mechanically verify large-scale
formalizations of critical components, including programming language
meta-theory~\cite{poplmark}, compilers~\cite{leroy09cc}, large
mathematical proofs~\cite{gonthier13engineering} and operating system
kernels~\cite{klein10seL4}. Due to their scale and complexity, these
developments can be quite time consuming, often demanding multiple
man-years of effort.

It is reasonable to expect that variations can simply
extend and reuse the original development in order to leverage the
large investment of resources in these formalizations.
This is unfortunately often not the case, as even small extensions
can require significant additional effort.
Adding a new language feature to a programming language formalization or
compiler, for example, involves significant redesigns that
have a cross-cutting impact on nearly all definitions and proofs in the
formalization. This leads to a copy-paste-patch approach to reuse with
several modified copies of the original development, making it difficult to
compose new features and ultimately leading to a maintenance nightmare.
%is desirable to structure their design as modularly as possible to
%enable easy modification and reuse. To this end, proof assistants such
%as Coq and Agda have powerful modularity constructs including modules,
%type classes and expressive forms of dependent types. Unfortunately,
%these modularity constructs are designed to support the addition of
%new definitions and proofs. When extension cut across these modularity
%boundaries by requiring a modification to existing definitions, as
%when adding a new piece of syntax to the expressions of a language,
%users are forced to manually edit the development, leading to a
%cut-paste-patch approach to reuse.
Dissatisfied with this situation, several
researchers~\cite{poplmark,shao10certified,stampoulis10VeriML,gonthier13engineering}
have called for better ways to modularize mechanical formalizations.

This work extends the current state-of-the-art in modular
mechanizations by solving a well-known and long-standing open problem
with denotational semantics: type soundness proofs are notoriously
brittle with respect to the addition of new effects. This is an
important problem because effects are pervasive in programming
language formalizations: in addition to extensions to syntax and
semantics, new features usually introduce new effects to the
denotations. Without a more robust formulation of type soundness, the
addition of new effects requires cross-cutting changes to type
soundness theorem statements and proofs.

Initially the semantics themselves were also brittle with respect to
effects~\cite{mosses84sdt,lee1989realistic}, but
\emph{monads}~\cite{Moggi89a,Wadler92a} have been found to provide the
necessary robustness to denotations. Yet as far as we know, the brittleness 
of (denotational) type soundness proofs has remained an open problem since it was raised by Wright and
Felleisen~\cite{wright94syntactic} to motivate their own type-soundness
approach. The framework we present here, \emph{modular monadic meta-theory} (\Name), is
the first in 20 years to provide a substantial solution. Using \Name, we develop a
novel approach to proving type soundness for monadic denotational
semantics in a way that is modular in the set of effects used. Proofs for
individual features do not depend on effects they do not use and hence
are robust to extension.

%%For example, it would be reasonable to expect that
%%after verifying a C compiler, verifying a C compiler with some
%%extensions requires a proportionaly equal effort to verify the language
%%extensions.
%Unfortunately, this is often not the case. Existing formal developments are
%usually \emph{monolithic} and extensions involve significant redesigns that
%have a pervasive impact on nearly all definitions and proofs in the
%formalization.

% \begin{figure}
% \begin{center}
% \includegraphics[scale=0.5]{src/modaxis.png}
% \end{center}
% \caption{Three dimensions of modularity.}
% \label{sec:axis}
% \vspace{-.5cm}
% \end{figure}

The solution builds on Meta-Theory \`{a} la Carte (MTC)~\cite{mtc}, a Coq
framework for the mechanization of formal programming language meta-theory that
supports modular extension of existing definitions. With MTC it is possible to
develop meta-theory which is modular in two dimensions: \emph{language features}
on the one hand and \emph{functions} and \emph{proofs} over these features on
the other hand. MTC adapts ideas from existing programming language
solutions~\cite{swierstra08dtc,oliveira09modular} to the \emph{expression
problem}~\cite{wadler98expression-problem} for functions and features, and adds
\emph{modular induction} for proofs.

\name adds a third modularity dimension to MTC: modular addition of new \emph{effects}. 
\name enables the separate definition of features with
effectful semantic functions and proofs over these functions, and reuse of
these features in formalizations of multiple languages.

To make denotations robust with respect to effects, \name uses the
established solution, \emph{monads}. In Coq, \emph{type
classes}~\cite{wadler89how-to} enable semantic function definitions
that are constrained, yet polymorphic in the monad. This allows the
inclusion of a feature in any language which supports a superset of
its effects. When a language is composed from different effectful
features, \emph{monad transformers}~\cite{liang95monad} are used to instantiate
the denotation's monad with all the effects required by the modular
components.

To solve the key challenge of modularizing and reusing theorems and proofs of 
type soundness, we split the classic type soundness theorems into three parts:
%%are highly challening because their traditional
%%formulations depend intimately
%%on the effects (such as state or exceptions) used by a language. As a
%%consequence, soundness statements over different sets of effects are
%%fundamentally incompatible and cannot be reused for supersets of
%%effects. \name overcomes this problem by splitting the problem into three
%%key theorems.
\begin{enumerate}
\item Reusable \emph{feature theorems} capture the essence
      of type soundness for an individual feature. 
      They depend only on that feature's syntax, typing relation, semantic function and the
      effects used therein. At the same time,
      they abstract over the syntax, semantics and effects of other features.
      This means that the addition of new features with other types of effects 
      \emph{does not affect} the existing feature theorem proofs. 

      To achieve the abstraction over other effects, a feature uses a constrained
      polymorphic monad. As a consequence, it only establishes the well-typing of the resulting
      denotations with respect to the effects declared in the constraints. 

\item Reusable \emph{effect theorems} fix the monad of denotations and consequently
      the set of effects. They take well-typing proofs of monadic denotations 
      expressed in terms of a constrained polymorphic monad and which mention only a subset of
      effects, and turn them into well-typings with respect to a fixed monad and all the effects it
      provides.
      
      Effect theorems reason fully at the level of denotations and
      abstract over the details of language features like syntax and semantic
      functions.

%       capture the overall interaction of
%       effects and evaluation for a particular combination of effects. Proofs
%       of this kind of theorem depend only on the set of effects used in
%       evaluation, and are orthogonal to features. Consequentely the same effect theorem
%       will work for any languages that use the particular combination of
%       effects captured by the theorem.



%%Proofs of this kind of theorem are independent
%%from any details of the features' semantic functions:
%%they only need to know that a feature preserves the reusable feature theorem.
\item Finally, \emph{language theorems} 
      establish type soundness for a particular language. 
      They require no more effort than to 
      instantiate the set of features and the set of effects (i.e., the monad), 
      thus tying together the respective feature and effect theorems into an overall proof.
\end{enumerate}
To establish the first two theorems, \name relies on modular induction and algebraic
laws about effects. As far as we know, it applies
the most comprehensive set of such laws to date, as
each effect utilized by a feature needs to
be backed up by laws and interactions between different
effects must also be governed by laws. These laws are crucial for
modular reasoning in the presence of effects.

% \BO{Expand and say something about case studies.}

\begin{comment}
An essential problem with existing development methodologies is that
they fail to deal with \emph{crosscutting concerns}. The problem is
not very different from problems which have been previously identified
in general software development.  As Tarr et al~\cite{f} argues most
programming languages tend to favor one form of decomposition, which
allows software to be modularized in certain way. Figure~\ref{sec:axis}
illustrates three different types of decomposition in which software
can be modularized. In the case of theorem provers like Coq, which can be viewed as functional
programming languages, the use of inductive datatypes allows
natural support for the modularization of \emph{operations} and \emph{proofs}.
However, as epitomized by the \emph{Expression Problem}~\cite{g}, if it is necessary
to extend the inductive datatypes with new cases then several
crosscutting changes are needed to update the operations and proofs.
Furthermore the introduction of effects in existing operations
requires pervasive changes to propagate effects in all
calling functions. In other words, while functional languages support
modularity naturally along the operations/proofs axis, they do not support
modularity well in the other two axis.


Previous work already provides three important ingredients towards
this goal: \emph{monads}~\cite{wadler}, \emph{modular
datatypes}~\cite{m} and \emph{modular induction}~\cite{e}. Monads are a
well-established way providing the semantics of languages with
effects; with the help of qualified types~\cite{j} and monad
transformers~\cite{k}, monads can be modularly composed. Modular
datatypes allow parts of a datatype and corresponding functions to be
defined separately and are helpful to localize particular features of
programming languages. Monads and modular datatypes have already
been shown to be an effective means to modularize

More recently it has been shown how to add
a modular form of induction on top of a variant of modular datatypes.
With modular induction it becomes possible to define

With monads and modular datatypes
semantic functions can be modularly defined for each language feature
without hard-wiring a particular set of effects or language
constructs~\cite{o}.

Previous work has already done significant progress towards
the development of modular techniques.

In previous work, we have proposed a solution for allowing modularity
in two dimensions simultaneously: operations/proofs and language constructs.
While for operations we were able, for the most part, to built on
existing solutions to the expression problem for mainstream languages,
dealing with proofs modularly required new techniques. In particular,
we had to develop new techniques which allowed us to express
\emph{inductive proofs} in a modular way. However, that work did
not provide a modular solution for effects. As a result, our meta-theory
framework was effectively limited to \emph{pure} languages.

In this work we provide a \emph{modular} solution for effects
that in combination with MTC allows the development of modular meta-theory
for languages with \emph{effects}. Much like our previous work,
dealing only with operations (not proofs) can, for the most part,
be done using existing techniques~\cite{j}. However, once again, the
key challenge lies in how to modularize theorems and proofs along the
three dimensions of modularity.
\end{comment}

\paragraph{}
In summary, the specific contributions of this work are:
\begin{itemize}

\item A reusable framework, \Name, for mechanized meta-theory of languages with effects.
This framework includes a mechanized library for monads, monad transformers and
corresponding algebraic laws in Coq. Besides several laws for
specific types of effects, the library also includes laws for
the interactions between different types of effects.

\item A new \emph{modular} proof method for type-soundness proofs of denotational semantics.

\item A case study of a family of fully mechanized languages, including
a mini-ML variant with errors and references. The case study comprises
28 languages, 8 different effect theorems and 5 features with their feature theorems.

\end{itemize}

\noindent\name is implemented in the Coq proof assistant and the code is available at
\url{http://www.cs.utexas.edu/~bendy/3MT}.
% Our implementation minimizes the
% programmer's burden for adding new features by automating the boilerplate with
% type classes and default tactics.
%%Moreover, the framework already provides modular
%%components for mini-ML as a starting point for new language formalizations.
% We also provide a complimentary Haskell
% implementation of the computational subset of code used in this paper.


%-------------------------------------------------------------------------------
\paragraph{Code and Notational Conventions}

While all the code underlying this paper has been developed in Coq,
the paper adopts a terser syntax for its many code fragments.  For the
computational parts, this syntax exactly coincides with Haskell
syntax, while it is an extrapolation of Haskell syntax style for
propositions and proof concepts. Following MTC, the Coq code requires the
impredicative-set option due to the use of Church encodings.


%% %% ODER: format ==         = "\mathrel{==}"
%% ODER: format /=         = "\neq "
%
%
\makeatletter
\@ifundefined{lhs2tex.lhs2tex.sty.read}%
  {\@namedef{lhs2tex.lhs2tex.sty.read}{}%
   \newcommand\SkipToFmtEnd{}%
   \newcommand\EndFmtInput{}%
   \long\def\SkipToFmtEnd#1\EndFmtInput{}%
  }\SkipToFmtEnd

\newcommand\ReadOnlyOnce[1]{\@ifundefined{#1}{\@namedef{#1}{}}\SkipToFmtEnd}
\usepackage{amstext}
\usepackage{amssymb}
\usepackage{stmaryrd}
\DeclareFontFamily{OT1}{cmtex}{}
\DeclareFontShape{OT1}{cmtex}{m}{n}
  {<5><6><7><8>cmtex8
   <9>cmtex9
   <10><10.95><12><14.4><17.28><20.74><24.88>cmtex10}{}
\DeclareFontShape{OT1}{cmtex}{m}{it}
  {<-> ssub * cmtt/m/it}{}
\newcommand{\texfamily}{\fontfamily{cmtex}\selectfont}
\DeclareFontShape{OT1}{cmtt}{bx}{n}
  {<5><6><7><8>cmtt8
   <9>cmbtt9
   <10><10.95><12><14.4><17.28><20.74><24.88>cmbtt10}{}
\DeclareFontShape{OT1}{cmtex}{bx}{n}
  {<-> ssub * cmtt/bx/n}{}
\newcommand{\tex}[1]{\text{\texfamily#1}}	% NEU

\newcommand{\Sp}{\hskip.33334em\relax}


\newcommand{\Conid}[1]{\mathit{#1}}
\newcommand{\Varid}[1]{\mathit{#1}}
\newcommand{\anonymous}{\kern0.06em \vbox{\hrule\@width.5em}}
\newcommand{\plus}{\mathbin{+\!\!\!+}}
\newcommand{\bind}{\mathbin{>\!\!\!>\mkern-6.7mu=}}
\newcommand{\rbind}{\mathbin{=\mkern-6.7mu<\!\!\!<}}% suggested by Neil Mitchell
\newcommand{\sequ}{\mathbin{>\!\!\!>}}
\renewcommand{\leq}{\leqslant}
\renewcommand{\geq}{\geqslant}
\usepackage{polytable}

%mathindent has to be defined
\@ifundefined{mathindent}%
  {\newdimen\mathindent\mathindent\leftmargini}%
  {}%

\def\resethooks{%
  \global\let\SaveRestoreHook\empty
  \global\let\ColumnHook\empty}
\newcommand*{\savecolumns}[1][default]%
  {\g@addto@macro\SaveRestoreHook{\savecolumns[#1]}}
\newcommand*{\restorecolumns}[1][default]%
  {\g@addto@macro\SaveRestoreHook{\restorecolumns[#1]}}
\newcommand*{\aligncolumn}[2]%
  {\g@addto@macro\ColumnHook{\column{#1}{#2}}}

\resethooks

\newcommand{\onelinecommentchars}{\quad-{}- }
\newcommand{\commentbeginchars}{\enskip\{-}
\newcommand{\commentendchars}{-\}\enskip}

\newcommand{\visiblecomments}{%
  \let\onelinecomment=\onelinecommentchars
  \let\commentbegin=\commentbeginchars
  \let\commentend=\commentendchars}

\newcommand{\invisiblecomments}{%
  \let\onelinecomment=\empty
  \let\commentbegin=\empty
  \let\commentend=\empty}

\visiblecomments

\newlength{\blanklineskip}
\setlength{\blanklineskip}{0.66084ex}

\newcommand{\hsindent}[1]{\quad}% default is fixed indentation
\let\hspre\empty
\let\hspost\empty
\newcommand{\NB}{\textbf{NB}}
\newcommand{\Todo}[1]{$\langle$\textbf{To do:}~#1$\rangle$}

\EndFmtInput
\makeatother
%
%
%
% First, let's redefine the forall, and the dot.
%
%
% This is made in such a way that after a forall, the next
% dot will be printed as a period, otherwise the formatting
% of `comp_` is used. By redefining `comp_`, as suitable
% composition operator can be chosen. Similarly, period_
% is used for the period.
%
\ReadOnlyOnce{forall.fmt}%
\makeatletter

% The HaskellResetHook is a list to which things can
% be added that reset the Haskell state to the beginning.
% This is to recover from states where the hacked intelligence
% is not sufficient.

\let\HaskellResetHook\empty
\newcommand*{\AtHaskellReset}[1]{%
  \g@addto@macro\HaskellResetHook{#1}}
\newcommand*{\HaskellReset}{\HaskellResetHook}

\global\let\hsforallread\empty

\newcommand\hsforall{\global\let\hsdot=\hsperiodonce}
\newcommand*\hsperiodonce[2]{#2\global\let\hsdot=\hscompose}
\newcommand*\hscompose[2]{#1}

\AtHaskellReset{\global\let\hsdot=\hscompose}

% In the beginning, we should reset Haskell once.
\HaskellReset

\makeatother
\EndFmtInput
%
%
%
%
%
% This package provides two environments suitable to take the place
% of hscode, called "plainhscode" and "arrayhscode". 
%
% The plain environment surrounds each code block by vertical space,
% and it uses \abovedisplayskip and \belowdisplayskip to get spacing
% similar to formulas. Note that if these dimensions are changed,
% the spacing around displayed math formulas changes as well.
% All code is indented using \leftskip.
%
% Changed 19.08.2004 to reflect changes in colorcode. Should work with
% CodeGroup.sty.
%
\ReadOnlyOnce{polycode.fmt}%
\makeatletter

\newcommand{\hsnewpar}[1]%
  {{\parskip=0pt\parindent=0pt\par\vskip #1\noindent}}

% can be used, for instance, to redefine the code size, by setting the
% command to \small or something alike
\newcommand{\hscodestyle}{}

% The command \sethscode can be used to switch the code formatting
% behaviour by mapping the hscode environment in the subst directive
% to a new LaTeX environment.

\newcommand{\sethscode}[1]%
  {\expandafter\let\expandafter\hscode\csname #1\endcsname
   \expandafter\let\expandafter\endhscode\csname end#1\endcsname}

% "compatibility" mode restores the non-polycode.fmt layout.

\newenvironment{compathscode}%
  {\par\noindent
   \advance\leftskip\mathindent
   \hscodestyle
   \let\\=\@normalcr
   \let\hspre\(\let\hspost\)%
   \pboxed}%
  {\endpboxed\)%
   \par\noindent
   \ignorespacesafterend}

\newcommand{\compaths}{\sethscode{compathscode}}

% "plain" mode is the proposed default.
% It should now work with \centering.
% This required some changes. The old version
% is still available for reference as oldplainhscode.

\newenvironment{plainhscode}%
  {\hsnewpar\abovedisplayskip
   \advance\leftskip\mathindent
   \hscodestyle
   \let\hspre\(\let\hspost\)%
   \pboxed}%
  {\endpboxed%
   \hsnewpar\belowdisplayskip
   \ignorespacesafterend}

\newenvironment{oldplainhscode}%
  {\hsnewpar\abovedisplayskip
   \advance\leftskip\mathindent
   \hscodestyle
   \let\\=\@normalcr
   \(\pboxed}%
  {\endpboxed\)%
   \hsnewpar\belowdisplayskip
   \ignorespacesafterend}

% Here, we make plainhscode the default environment.

\newcommand{\plainhs}{\sethscode{plainhscode}}
\newcommand{\oldplainhs}{\sethscode{oldplainhscode}}
\plainhs

% The arrayhscode is like plain, but makes use of polytable's
% parray environment which disallows page breaks in code blocks.

\newenvironment{arrayhscode}%
  {\hsnewpar\abovedisplayskip
   \advance\leftskip\mathindent
   \hscodestyle
   \let\\=\@normalcr
   \(\parray}%
  {\endparray\)%
   \hsnewpar\belowdisplayskip
   \ignorespacesafterend}

\newcommand{\arrayhs}{\sethscode{arrayhscode}}

% The mathhscode environment also makes use of polytable's parray 
% environment. It is supposed to be used only inside math mode 
% (I used it to typeset the type rules in my thesis).

\newenvironment{mathhscode}%
  {\parray}{\endparray}

\newcommand{\mathhs}{\sethscode{mathhscode}}

% texths is similar to mathhs, but works in text mode.

\newenvironment{texthscode}%
  {\(\parray}{\endparray\)}

\newcommand{\texths}{\sethscode{texthscode}}

% The framed environment places code in a framed box.

\def\codeframewidth{\arrayrulewidth}
\RequirePackage{calc}

\newenvironment{framedhscode}%
  {\parskip=\abovedisplayskip\par\noindent
   \hscodestyle
   \arrayrulewidth=\codeframewidth
   \tabular{@{}|p{\linewidth-2\arraycolsep-2\arrayrulewidth-2pt}|@{}}%
   \hline\framedhslinecorrect\\{-1.5ex}%
   \let\endoflinesave=\\
   \let\\=\@normalcr
   \(\pboxed}%
  {\endpboxed\)%
   \framedhslinecorrect\endoflinesave{.5ex}\hline
   \endtabular
   \parskip=\belowdisplayskip\par\noindent
   \ignorespacesafterend}

\newcommand{\framedhslinecorrect}[2]%
  {#1[#2]}

\newcommand{\framedhs}{\sethscode{framedhscode}}

% The inlinehscode environment is an experimental environment
% that can be used to typeset displayed code inline.

\newenvironment{inlinehscode}%
  {\(\def\column##1##2{}%
   \let\>\undefined\let\<\undefined\let\\\undefined
   \newcommand\>[1][]{}\newcommand\<[1][]{}\newcommand\\[1][]{}%
   \def\fromto##1##2##3{##3}%
   \def\nextline{}}{\) }%

\newcommand{\inlinehs}{\sethscode{inlinehscode}}

% The joincode environment is a separate environment that
% can be used to surround and thereby connect multiple code
% blocks.

\newenvironment{joincode}%
  {\let\orighscode=\hscode
   \let\origendhscode=\endhscode
   \def\endhscode{\def\hscode{\endgroup\def\@currenvir{hscode}\\}\begingroup}
   %\let\SaveRestoreHook=\empty
   %\let\ColumnHook=\empty
   %\let\resethooks=\empty
   \orighscode\def\hscode{\endgroup\def\@currenvir{hscode}}}%
  {\origendhscode
   \global\let\hscode=\orighscode
   \global\let\endhscode=\origendhscode}%

\makeatother
\EndFmtInput
%

\section{Introduction}
\label{sec:intro}

Theorem provers are actively used to mechanically verify large-scale
formalizations of critical components, including programming language
meta-theory~\cite{poplmark}, compilers~\cite{leroy09cc}, large
mathematical proofs~\cite{gonthier13engineering} and operating system
kernels~\cite{klein10seL4}. Due to their scale and complexity, these
developments can be quite time consuming, often demanding multiple
man-years of effort.

It is reasonable to expect that variations can simply
extend and reuse the original development in order to leverage the
large investment of resources in these formalizations.
This is unfortunately often not the case, as even small extensions
can require significant additional effort.
Adding a new language feature to a programming language formalization or
compiler, for example, involves significant redesigns that
have a cross-cutting impact on nearly all definitions and proofs in the
formalization. This leads to a copy-paste-patch approach to reuse with
several modified copies of the original development, making it difficult to
compose new features and ultimately leading to a maintenance nightmare.
%is desirable to structure their design as modularly as possible to
%enable easy modification and reuse. To this end, proof assistants such
%as Coq and Agda have powerful modularity constructs including modules,
%type classes and expressive forms of dependent types. Unfortunately,
%these modularity constructs are designed to support the addition of
%new definitions and proofs. When extension cut across these modularity
%boundaries by requiring a modification to existing definitions, as
%when adding a new piece of syntax to the expressions of a language,
%users are forced to manually edit the development, leading to a
%cut-paste-patch approach to reuse.
Dissatisfied with this situation, several
researchers~\cite{poplmark,shao10certified,stampoulis10VeriML,gonthier13engineering}
have called for better ways to modularize mechanical formalizations.

This work extends the current state-of-the-art in modular
mechanizations by solving a well-known and long-standing open problem
with denotational semantics: type soundness proofs are notoriously
brittle with respect to the addition of new effects. This is an
important problem because effects are pervasive in programming
language formalizations: in addition to extensions to syntax and
semantics, new features usually introduce new effects to the
denotations. Without a more robust formulation of type soundness, the
addition of new effects requires cross-cutting changes to type
soundness theorem statements and proofs.

Initially the semantics themselves were also brittle with respect to
effects~\cite{mosses84sdt,lee1989realistic}, but
\emph{monads}~\cite{Moggi89a,Wadler92a} have been found to provide the
necessary robustness to denotations. Yet as far as we know, the brittleness 
of (denotational) type soundness proofs has remained an open problem since it was raised by Wright and
Felleisen~\cite{wright94syntactic} to motivate their own type-soundness
approach. The framework we present here, \emph{modular monadic meta-theory} (\Name), is
the first in 20 years to provide a substantial solution. Using \Name, we develop a
novel approach to proving type soundness for monadic denotational
semantics in a way that is modular in the set of effects used. Proofs for
individual features do not depend on effects they do not use and hence
are robust to extension.

%%For example, it would be reasonable to expect that
%%after verifying a C compiler, verifying a C compiler with some
%%extensions requires a proportionaly equal effort to verify the language
%%extensions.
%Unfortunately, this is often not the case. Existing formal developments are
%usually \emph{monolithic} and extensions involve significant redesigns that
%have a pervasive impact on nearly all definitions and proofs in the
%formalization.

% \begin{figure}
% \begin{center}
% \includegraphics[scale=0.5]{src/modaxis.png}
% \end{center}
% \caption{Three dimensions of modularity.}
% \label{sec:axis}
% \vspace{-.5cm}
% \end{figure}

The solution builds on Meta-Theory \`{a} la Carte (MTC)~\cite{mtc}, a Coq
framework for the mechanization of formal programming language meta-theory that
supports modular extension of existing definitions. With MTC it is possible to
develop meta-theory which is modular in two dimensions: \emph{language features}
on the one hand and \emph{functions} and \emph{proofs} over these features on
the other hand. MTC adapts ideas from existing programming language
solutions~\cite{swierstra08dtc,oliveira09modular} to the \emph{expression
problem}~\cite{wadler98expression-problem} for functions and features, and adds
\emph{modular induction} for proofs.

\name adds a third modularity dimension to MTC: modular addition of new \emph{effects}. 
\name enables the separate definition of features with
effectful semantic functions and proofs over these functions, and reuse of
these features in formalizations of multiple languages.

To make denotations robust with respect to effects, \name uses the
established solution, \emph{monads}. In Coq, \emph{type
classes}~\cite{wadler89how-to} enable semantic function definitions
that are constrained, yet polymorphic in the monad. This allows the
inclusion of a feature in any language which supports a superset of
its effects. When a language is composed from different effectful
features, \emph{monad transformers}~\cite{liang95monad} are used to instantiate
the denotation's monad with all the effects required by the modular
components.

To solve the key challenge of modularizing and reusing theorems and proofs of 
type soundness, we split the classic type soundness theorems into three parts:
%%are highly challening because their traditional
%%formulations depend intimately
%%on the effects (such as state or exceptions) used by a language. As a
%%consequence, soundness statements over different sets of effects are
%%fundamentally incompatible and cannot be reused for supersets of
%%effects. \name overcomes this problem by splitting the problem into three
%%key theorems.
\begin{enumerate}
\item Reusable \emph{feature theorems} capture the essence
      of type soundness for an individual feature. 
      They depend only on that feature's syntax, typing relation, semantic function and the
      effects used therein. At the same time,
      they abstract over the syntax, semantics and effects of other features.
      This means that the addition of new features with other types of effects 
      \emph{does not affect} the existing feature theorem proofs. 

      To achieve the abstraction over other effects, a feature uses a constrained
      polymorphic monad. As a consequence, it only establishes the well-typing of the resulting
      denotations with respect to the effects declared in the constraints. 

\item Reusable \emph{effect theorems} fix the monad of denotations and consequently
      the set of effects. They take well-typing proofs of monadic denotations 
      expressed in terms of a constrained polymorphic monad and which mention only a subset of
      effects, and turn them into well-typings with respect to a fixed monad and all the effects it
      provides.
      
      Effect theorems reason fully at the level of denotations and
      abstract over the details of language features like syntax and semantic
      functions.

%       capture the overall interaction of
%       effects and evaluation for a particular combination of effects. Proofs
%       of this kind of theorem depend only on the set of effects used in
%       evaluation, and are orthogonal to features. Consequentely the same effect theorem
%       will work for any languages that use the particular combination of
%       effects captured by the theorem.



%%Proofs of this kind of theorem are independent
%%from any details of the features' semantic functions:
%%they only need to know that a feature preserves the reusable feature theorem.
\item Finally, \emph{language theorems} 
      establish type soundness for a particular language. 
      They require no more effort than to 
      instantiate the set of features and the set of effects (i.e., the monad), 
      thus tying together the respective feature and effect theorems into an overall proof.
\end{enumerate}
To establish the first two theorems, \name relies on modular induction and algebraic
laws about effects. As far as we know, it applies
the most comprehensive set of such laws to date, as
each effect utilized by a feature needs to
be backed up by laws and interactions between different
effects must also be governed by laws. These laws are crucial for
modular reasoning in the presence of effects.

% \BO{Expand and say something about case studies.}

\begin{comment}
An essential problem with existing development methodologies is that
they fail to deal with \emph{crosscutting concerns}. The problem is
not very different from problems which have been previously identified
in general software development.  As Tarr et al~\cite{f} argues most
programming languages tend to favor one form of decomposition, which
allows software to be modularized in certain way. Figure~\ref{sec:axis}
illustrates three different types of decomposition in which software
can be modularized. In the case of theorem provers like Coq, which can be viewed as functional
programming languages, the use of inductive datatypes allows
natural support for the modularization of \emph{operations} and \emph{proofs}.
However, as epitomized by the \emph{Expression Problem}~\cite{g}, if it is necessary
to extend the inductive datatypes with new cases then several
crosscutting changes are needed to update the operations and proofs.
Furthermore the introduction of effects in existing operations
requires pervasive changes to propagate effects in all
calling functions. In other words, while functional languages support
modularity naturally along the operations/proofs axis, they do not support
modularity well in the other two axis.


Previous work already provides three important ingredients towards
this goal: \emph{monads}~\cite{wadler}, \emph{modular
datatypes}~\cite{m} and \emph{modular induction}~\cite{e}. Monads are a
well-established way providing the semantics of languages with
effects; with the help of qualified types~\cite{j} and monad
transformers~\cite{k}, monads can be modularly composed. Modular
datatypes allow parts of a datatype and corresponding functions to be
defined separately and are helpful to localize particular features of
programming languages. Monads and modular datatypes have already
been shown to be an effective means to modularize

More recently it has been shown how to add
a modular form of induction on top of a variant of modular datatypes.
With modular induction it becomes possible to define

With monads and modular datatypes
semantic functions can be modularly defined for each language feature
without hard-wiring a particular set of effects or language
constructs~\cite{o}.

Previous work has already done significant progress towards
the development of modular techniques.

In previous work, we have proposed a solution for allowing modularity
in two dimensions simultaneously: operations/proofs and language constructs.
While for operations we were able, for the most part, to built on
existing solutions to the expression problem for mainstream languages,
dealing with proofs modularly required new techniques. In particular,
we had to develop new techniques which allowed us to express
\emph{inductive proofs} in a modular way. However, that work did
not provide a modular solution for effects. As a result, our meta-theory
framework was effectively limited to \emph{pure} languages.

In this work we provide a \emph{modular} solution for effects
that in combination with MTC allows the development of modular meta-theory
for languages with \emph{effects}. Much like our previous work,
dealing only with operations (not proofs) can, for the most part,
be done using existing techniques~\cite{j}. However, once again, the
key challenge lies in how to modularize theorems and proofs along the
three dimensions of modularity.
\end{comment}

\paragraph{}
In summary, the specific contributions of this work are:
\begin{itemize}

\item A reusable framework, \Name, for mechanized meta-theory of languages with effects.
This framework includes a mechanized library for monads, monad transformers and
corresponding algebraic laws in Coq. Besides several laws for
specific types of effects, the library also includes laws for
the interactions between different types of effects.

\item A new \emph{modular} proof method for type-soundness proofs of denotational semantics.

\item A case study of a family of fully mechanized languages, including
a mini-ML variant with errors and references. The case study comprises
28 languages, 8 different effect theorems and 5 features with their feature theorems.

\end{itemize}

\noindent\name is implemented in the Coq proof assistant and the code is available at
\url{http://www.cs.utexas.edu/~bendy/3MT}.
% Our implementation minimizes the
% programmer's burden for adding new features by automating the boilerplate with
% type classes and default tactics.
%%Moreover, the framework already provides modular
%%components for mini-ML as a starting point for new language formalizations.
% We also provide a complimentary Haskell
% implementation of the computational subset of code used in this paper.


%-------------------------------------------------------------------------------
\paragraph{Code and Notational Conventions}

While all the code underlying this paper has been developed in Coq,
the paper adopts a terser syntax for its many code fragments.  For the
computational parts, this syntax exactly coincides with Haskell
syntax, while it is an extrapolation of Haskell syntax style for
propositions and proof concepts. Following MTC, the Coq code requires the
impredicative-set option due to the use of Church encodings.


%% \clearpage
\input{src/GenBinding/Overview}


\chapter{The \Knot Specification Language}\label{ch:knotsyntax}
\input{src/GenBinding/Specification}

\chapter{Semantics}\label{ch:knotsemantics}

The previous chapter has introduced the \Knot~specification language for
abstract syntax. This chapter generically defines the semantics of \Knot in
terms of a de Bruijn representation. We assume a given and fixed well-formed
specification $\spec$ in the rest of this chapter. The discussion follows these
consecutive steps:

Section \ref{sec:semantics} declares \emph{de Bruijn syntax terms} for \Knot's
\emph{abstract syntax declarations} of the given specification and specifies
which are well-sorted and well-scoped with respect to the specification. It also
defines the \emph{semantics of binding specifications} by means of evaluation.
\emph{Shifting and substitution functions} operating on terms are defined in
Section \ref{sec:operation}. This is necessary boilerplate that is needed to
define \emph{evaluation of expressions} which is the topic of Section
\ref{sec:exprsemantics}. The \emph{interpretation of relation specifications} is
discussed in Section \ref{sec:relationsemantics}.

\input{src/GenBinding/TermSemantics}
\input{src/GenBinding/Operations}
\input{src/GenBinding/RelationSemantics}

\chapter{Boilerplate Lemmas}\label{ch:elaboration}
\input{src/GenBinding/WellScopedness}
\input{src/GenBinding/RelShifting}
\input{src/GenBinding/RelSubstitution}

\chapter{Code generation}\label{ch:codegen}
\input{src/GenBinding/Generation}

\chapter{Evaluation}
%% ODER: format ==         = "\mathrel{==}"
%% ODER: format /=         = "\neq "
%
%
\makeatletter
\@ifundefined{lhs2tex.lhs2tex.sty.read}%
  {\@namedef{lhs2tex.lhs2tex.sty.read}{}%
   \newcommand\SkipToFmtEnd{}%
   \newcommand\EndFmtInput{}%
   \long\def\SkipToFmtEnd#1\EndFmtInput{}%
  }\SkipToFmtEnd

\newcommand\ReadOnlyOnce[1]{\@ifundefined{#1}{\@namedef{#1}{}}\SkipToFmtEnd}
\usepackage{amstext}
\usepackage{amssymb}
\usepackage{stmaryrd}
\DeclareFontFamily{OT1}{cmtex}{}
\DeclareFontShape{OT1}{cmtex}{m}{n}
  {<5><6><7><8>cmtex8
   <9>cmtex9
   <10><10.95><12><14.4><17.28><20.74><24.88>cmtex10}{}
\DeclareFontShape{OT1}{cmtex}{m}{it}
  {<-> ssub * cmtt/m/it}{}
\newcommand{\texfamily}{\fontfamily{cmtex}\selectfont}
\DeclareFontShape{OT1}{cmtt}{bx}{n}
  {<5><6><7><8>cmtt8
   <9>cmbtt9
   <10><10.95><12><14.4><17.28><20.74><24.88>cmbtt10}{}
\DeclareFontShape{OT1}{cmtex}{bx}{n}
  {<-> ssub * cmtt/bx/n}{}
\newcommand{\tex}[1]{\text{\texfamily#1}}	% NEU

\newcommand{\Sp}{\hskip.33334em\relax}


\newcommand{\Conid}[1]{\mathit{#1}}
\newcommand{\Varid}[1]{\mathit{#1}}
\newcommand{\anonymous}{\kern0.06em \vbox{\hrule\@width.5em}}
\newcommand{\plus}{\mathbin{+\!\!\!+}}
\newcommand{\bind}{\mathbin{>\!\!\!>\mkern-6.7mu=}}
\newcommand{\rbind}{\mathbin{=\mkern-6.7mu<\!\!\!<}}% suggested by Neil Mitchell
\newcommand{\sequ}{\mathbin{>\!\!\!>}}
\renewcommand{\leq}{\leqslant}
\renewcommand{\geq}{\geqslant}
\usepackage{polytable}

%mathindent has to be defined
\@ifundefined{mathindent}%
  {\newdimen\mathindent\mathindent\leftmargini}%
  {}%

\def\resethooks{%
  \global\let\SaveRestoreHook\empty
  \global\let\ColumnHook\empty}
\newcommand*{\savecolumns}[1][default]%
  {\g@addto@macro\SaveRestoreHook{\savecolumns[#1]}}
\newcommand*{\restorecolumns}[1][default]%
  {\g@addto@macro\SaveRestoreHook{\restorecolumns[#1]}}
\newcommand*{\aligncolumn}[2]%
  {\g@addto@macro\ColumnHook{\column{#1}{#2}}}

\resethooks

\newcommand{\onelinecommentchars}{\quad-{}- }
\newcommand{\commentbeginchars}{\enskip\{-}
\newcommand{\commentendchars}{-\}\enskip}

\newcommand{\visiblecomments}{%
  \let\onelinecomment=\onelinecommentchars
  \let\commentbegin=\commentbeginchars
  \let\commentend=\commentendchars}

\newcommand{\invisiblecomments}{%
  \let\onelinecomment=\empty
  \let\commentbegin=\empty
  \let\commentend=\empty}

\visiblecomments

\newlength{\blanklineskip}
\setlength{\blanklineskip}{0.66084ex}

\newcommand{\hsindent}[1]{\quad}% default is fixed indentation
\let\hspre\empty
\let\hspost\empty
\newcommand{\NB}{\textbf{NB}}
\newcommand{\Todo}[1]{$\langle$\textbf{To do:}~#1$\rangle$}

\EndFmtInput
\makeatother
%
%
%
% First, let's redefine the forall, and the dot.
%
%
% This is made in such a way that after a forall, the next
% dot will be printed as a period, otherwise the formatting
% of `comp_` is used. By redefining `comp_`, as suitable
% composition operator can be chosen. Similarly, period_
% is used for the period.
%
\ReadOnlyOnce{forall.fmt}%
\makeatletter

% The HaskellResetHook is a list to which things can
% be added that reset the Haskell state to the beginning.
% This is to recover from states where the hacked intelligence
% is not sufficient.

\let\HaskellResetHook\empty
\newcommand*{\AtHaskellReset}[1]{%
  \g@addto@macro\HaskellResetHook{#1}}
\newcommand*{\HaskellReset}{\HaskellResetHook}

\global\let\hsforallread\empty

\newcommand\hsforall{\global\let\hsdot=\hsperiodonce}
\newcommand*\hsperiodonce[2]{#2\global\let\hsdot=\hscompose}
\newcommand*\hscompose[2]{#1}

\AtHaskellReset{\global\let\hsdot=\hscompose}

% In the beginning, we should reset Haskell once.
\HaskellReset

\makeatother
\EndFmtInput
%
%
%
%
%
% This package provides two environments suitable to take the place
% of hscode, called "plainhscode" and "arrayhscode". 
%
% The plain environment surrounds each code block by vertical space,
% and it uses \abovedisplayskip and \belowdisplayskip to get spacing
% similar to formulas. Note that if these dimensions are changed,
% the spacing around displayed math formulas changes as well.
% All code is indented using \leftskip.
%
% Changed 19.08.2004 to reflect changes in colorcode. Should work with
% CodeGroup.sty.
%
\ReadOnlyOnce{polycode.fmt}%
\makeatletter

\newcommand{\hsnewpar}[1]%
  {{\parskip=0pt\parindent=0pt\par\vskip #1\noindent}}

% can be used, for instance, to redefine the code size, by setting the
% command to \small or something alike
\newcommand{\hscodestyle}{}

% The command \sethscode can be used to switch the code formatting
% behaviour by mapping the hscode environment in the subst directive
% to a new LaTeX environment.

\newcommand{\sethscode}[1]%
  {\expandafter\let\expandafter\hscode\csname #1\endcsname
   \expandafter\let\expandafter\endhscode\csname end#1\endcsname}

% "compatibility" mode restores the non-polycode.fmt layout.

\newenvironment{compathscode}%
  {\par\noindent
   \advance\leftskip\mathindent
   \hscodestyle
   \let\\=\@normalcr
   \let\hspre\(\let\hspost\)%
   \pboxed}%
  {\endpboxed\)%
   \par\noindent
   \ignorespacesafterend}

\newcommand{\compaths}{\sethscode{compathscode}}

% "plain" mode is the proposed default.
% It should now work with \centering.
% This required some changes. The old version
% is still available for reference as oldplainhscode.

\newenvironment{plainhscode}%
  {\hsnewpar\abovedisplayskip
   \advance\leftskip\mathindent
   \hscodestyle
   \let\hspre\(\let\hspost\)%
   \pboxed}%
  {\endpboxed%
   \hsnewpar\belowdisplayskip
   \ignorespacesafterend}

\newenvironment{oldplainhscode}%
  {\hsnewpar\abovedisplayskip
   \advance\leftskip\mathindent
   \hscodestyle
   \let\\=\@normalcr
   \(\pboxed}%
  {\endpboxed\)%
   \hsnewpar\belowdisplayskip
   \ignorespacesafterend}

% Here, we make plainhscode the default environment.

\newcommand{\plainhs}{\sethscode{plainhscode}}
\newcommand{\oldplainhs}{\sethscode{oldplainhscode}}
\plainhs

% The arrayhscode is like plain, but makes use of polytable's
% parray environment which disallows page breaks in code blocks.

\newenvironment{arrayhscode}%
  {\hsnewpar\abovedisplayskip
   \advance\leftskip\mathindent
   \hscodestyle
   \let\\=\@normalcr
   \(\parray}%
  {\endparray\)%
   \hsnewpar\belowdisplayskip
   \ignorespacesafterend}

\newcommand{\arrayhs}{\sethscode{arrayhscode}}

% The mathhscode environment also makes use of polytable's parray 
% environment. It is supposed to be used only inside math mode 
% (I used it to typeset the type rules in my thesis).

\newenvironment{mathhscode}%
  {\parray}{\endparray}

\newcommand{\mathhs}{\sethscode{mathhscode}}

% texths is similar to mathhs, but works in text mode.

\newenvironment{texthscode}%
  {\(\parray}{\endparray\)}

\newcommand{\texths}{\sethscode{texthscode}}

% The framed environment places code in a framed box.

\def\codeframewidth{\arrayrulewidth}
\RequirePackage{calc}

\newenvironment{framedhscode}%
  {\parskip=\abovedisplayskip\par\noindent
   \hscodestyle
   \arrayrulewidth=\codeframewidth
   \tabular{@{}|p{\linewidth-2\arraycolsep-2\arrayrulewidth-2pt}|@{}}%
   \hline\framedhslinecorrect\\{-1.5ex}%
   \let\endoflinesave=\\
   \let\\=\@normalcr
   \(\pboxed}%
  {\endpboxed\)%
   \framedhslinecorrect\endoflinesave{.5ex}\hline
   \endtabular
   \parskip=\belowdisplayskip\par\noindent
   \ignorespacesafterend}

\newcommand{\framedhslinecorrect}[2]%
  {#1[#2]}

\newcommand{\framedhs}{\sethscode{framedhscode}}

% The inlinehscode environment is an experimental environment
% that can be used to typeset displayed code inline.

\newenvironment{inlinehscode}%
  {\(\def\column##1##2{}%
   \let\>\undefined\let\<\undefined\let\\\undefined
   \newcommand\>[1][]{}\newcommand\<[1][]{}\newcommand\\[1][]{}%
   \def\fromto##1##2##3{##3}%
   \def\nextline{}}{\) }%

\newcommand{\inlinehs}{\sethscode{inlinehscode}}

% The joincode environment is a separate environment that
% can be used to surround and thereby connect multiple code
% blocks.

\newenvironment{joincode}%
  {\let\orighscode=\hscode
   \let\origendhscode=\endhscode
   \def\endhscode{\def\hscode{\endgroup\def\@currenvir{hscode}\\}\begingroup}
   %\let\SaveRestoreHook=\empty
   %\let\ColumnHook=\empty
   %\let\resethooks=\empty
   \orighscode\def\hscode{\endgroup\def\@currenvir{hscode}}}%
  {\origendhscode
   \global\let\hscode=\orighscode
   \global\let\endhscode=\origendhscode}%

\makeatother
\EndFmtInput
%

% format runC         = "run\mathbb{C}_{\scalebox{0.6}{T}}"
% format C            = "\mathbb{C}_{\scalebox{0.6}{T}}"
%%format R            = "\mathbb{R}_{\scalebox{0.6}{T}}"

















%%format ._ = "."







%
%
\ReadOnlyOnce{exists.fmt}%
\makeatletter

\newcommand\hsexists{\global\let\hsdot=\hsperiodonce}

\AtHaskellReset{\global\let\hsdot=\hscompose}

% In the beginning, we should reset Haskell once.
\HaskellReset

\makeatother
\EndFmtInput

\section{Case study}
\label{sec:casestudy}

As a demonstration of the advantages of our approach over MTC's
Church-encoding based approach, we have ported the case study from
\cite{mtc}.

The study consists of five reusable language features with soundness
and continuity\footnote{of step-bounded evaluation functions} proofs
in addition to typing and evaluation functions.
Figure~\ref{fig:MiniMLSyntax} presents the syntax of the expressions,
values, and types provided by the features; each line is annotated
with the feature that provides that set of definitions.

\begin{figure}[t]
  \begin{center}
    \begin{minipage}{\columnwidth}
      \begin{center}
        \fbox{
        \hspace{-.3cm}
          \begin{tabular}{r@{~}c@{~}lr}
            {\tt e} & ::= & {\tt $\mathbb{N}$ $|$ e + e} & \textit{Arith}\\
            & $|$ &  {\tt $\mathbb{B}$ $|$ {\bf if} e {\bf then} e {\bf else} e} & \textit{Bool} \\
           %% & $|$ & {\tt {\bf case} e {\bf of \{ z} $\Rightarrow$ e $\mathbf{;}$ {\bf S} n $\Rightarrow$ e\}} & \textit{NatCase}\\
            & $|$ & {\tt {\bf lam} x : T.e $|$ e e $|$ x} & \textit{Lambda}\\
            & $|$ & {\tt {\bf ref} e $|$ !e $|$ e:=e} & \textit{References}\\
            & $|$ & {\tt {\bf try} e {\bf with} e} $|$ {\bf error} & \textit{Errors}\\
           %%  & $|$ & {\tt {\bf fix} x : T.e} & \textit{Recursion}\\
          \end{tabular}
        }
      \end{center}
    \end{minipage}
    \begin{minipage}{\columnwidth}
      \hspace{-.3cm}
    \begin{tabular}{cc}
      \begin{minipage}{.48\columnwidth}
        \fbox{
        \hspace{-.3cm}
          \begin{tabular}{r@{~}c@{~}lr}
            {\tt V} & ::= & {\tt $\mathbb{N}$} & \textit{Arith}\\
            & $|$ &  {\tt $\mathbb{B}$} & \textit{Bool} \\
            & $|$ & {\tt {\bf clos} e $\mathtt{\overline{V}}$} & \textit{Lambda}\\
            & $|$ & {\tt {\bf loc} $\mathbb{N}$} & \textit{References}\\
          \end{tabular}
        }
      \end{minipage} &
      \begin{minipage}{.38\columnwidth}
        \hspace{-.3cm}
        \fbox{
        \hspace{-.3cm}
          \begin{tabular}{r@{~}c@{~}lr}
            {\tt T} & ::= & {\tt \bf Nat} & \textit{Arith}\\
            & $|$ &  {\tt \bf Bool} & \textit{Bool} \\
            & $|$ & {\tt T $\rightarrow$ T} & \textit{Lambda}\\
            & $|$ & {\tt {\bf Ref} T} & \textit{References}\\
          \end{tabular}
        }
      \end{minipage}
    \end{tabular}
    \end{minipage}
  \end{center}
  \caption{mini-ML expressions, values, and types}
  \label{fig:MiniMLSyntax}
\end{figure}


In this section we discuss the benefits and trade-offs we have
experienced while porting the case study to our approach.

\paragraph{Code size}

By the move to a datatype-generic approach the underlying modular
framework grew from about 2500 LoC to about 3500 LoC. This includes
both the universe of containers and polynomial functors and the
generic implementations of fold, induction and equality.

The size of the implementation of the modular feature components is
roughly 1100 LoC per feature in the original MTC case study. By
switching from Church-encodings to a datatype-generic approach we
stripped away on average 70 LoC of additional proof obligations needed
for reasoning with Church-encodings per feature. However, instantiating
the MTC interface amounts to roughly 40 LoC per feature while our approach
requires about 100 LoC per feature for the instances.

By using the generic equality and generic proofs about its properties
we can remove the specific implementations from the case study. This
is about 40 LoC per feature. In total we could reduce the average size
of a feature implementation to 1050 LoC.

\paragraph{Impredicative sets}\label{ssec:impredicativeset}

The higher-rank type in the definition of \ensuremath{\Varid{Fix_{M}}}
\begin{hscode}\SaveRestoreHook
\column{B}{@{}>{\hspre}l<{\hspost}@{}}%
\column{3}{@{}>{\hspre}l<{\hspost}@{}}%
\column{E}{@{}>{\hspre}l<{\hspost}@{}}%
\>[3]{}\Varid{Fix_{M}}\;(\Varid{f}\mathbin{::}\Conid{Set}\to \Conid{Set})\mathrel{=}\forall (\Varid{a}\mathbin{::}\Conid{Set})\hsforall \hsdot{\circ }{.}\Conid{Algebra}\;\Varid{f}\;\Varid{a}\to \Varid{a}{}\<[E]%
\ColumnHook
\end{hscode}\resethooks
causes \ensuremath{\Varid{Fix_{M}}\;\Varid{f}} to be in a higher universe level than the domain of
\ensuremath{\Varid{f}}. Hence to use \ensuremath{\Varid{Fix_{M}}\;\Varid{f}} as a fixpoint of \ensuremath{\Varid{f}} we need an
impredicative sort. MTC uses Coq's impredicative-set option for this
which is known to lead to logical inconsistencies.

When constructing the fixpoint of a container we do not need to raise
the universe level and can thus avoid using impredicative sets.

\paragraph{Induction principles}


Church encodings have problems supporting proper induction principles,
like the induction principle for arithmetic expressions \ensuremath{\Varid{ind}_{\Conid{A}}} in
Section \ref{ssec:modularinductivereasoning}. MTC uses a
\emph{poor-man's induction principle} \ensuremath{\Varid{ind}_{\Conid{A}}^2} instead.
\begin{hscode}\SaveRestoreHook
\column{B}{@{}>{\hspre}l<{\hspost}@{}}%
\column{3}{@{}>{\hspre}l<{\hspost}@{}}%
\column{5}{@{}>{\hspre}l<{\hspost}@{}}%
\column{18}{@{}>{\hspre}l<{\hspost}@{}}%
\column{E}{@{}>{\hspre}l<{\hspost}@{}}%
\>[3]{}\Varid{ind}_{\Conid{A}}^2\mathbin{::}{}\<[E]%
\\
\>[3]{}\hsindent{2}{}\<[5]%
\>[5]{}\forall (\Varid{p}{}\<[18]%
\>[18]{}\mathbin{::}(\Conid{Arith}\to \Conid{Prop})).{}\<[E]%
\\
\>[3]{}\hsindent{2}{}\<[5]%
\>[5]{}\forall (\Varid{hl}{}\<[18]%
\>[18]{}\mathbin{::}(\forall \Varid{n}\hsforall .\Varid{p}\;(\Conid{InMTC}\;(\Conid{Lit}\;\Varid{n})))).{}\<[E]%
\\
\>[3]{}\hsindent{2}{}\<[5]%
\>[5]{}\forall (\Varid{ha}{}\<[18]%
\>[18]{}\mathbin{::}(\forall \Varid{x}\hsforall \;\Varid{y}.\Varid{p}\;\Varid{x}\to \Varid{p}\;\Varid{y}\to \Varid{p}\;(\Conid{InMTC}\;(\Conid{Add}\;\Varid{x}\;\Varid{y})))).{}\<[E]%
\\
\>[3]{}\hsindent{2}{}\<[5]%
\>[5]{}\Conid{Algebra}\;\Varid{Arith}_F\;(\exists \Varid{a}\hsexists .\Varid{p}\;\Varid{a})\hsforall \hsforall \hsforall {}\<[E]%
\ColumnHook
\end{hscode}\resethooks
The induction principle uses a dependent sum type to turn a proof
algebras into a regular algebra. The algebra builds a copy of the
original term and a proof that the property holds for the copy. The
proof for the copy can be obtained by folding with this algebra. In
order to draw conclusions about the original term two additional
\emph{well-formedness} conditions have to be proven.
\begin{enumerate}

\item The proof-algebra has to be well-formed in the sense that it
really builds a copy of the original term instead of producing an
arbitrary term. This proof needs to be done only once for every
induction principle of every functor and is about 20 LoC per
feature. The use of this well-formedness proof is completely automated
using type-classes and hence hidden from the user.

\item The fold operator used to build the proof using the algebra
needs to be a proper fold operator, i.e. it needs to satisfy the
universal property of folds.
\begin{hscode}\SaveRestoreHook
\column{B}{@{}>{\hspre}l<{\hspost}@{}}%
\column{3}{@{}>{\hspre}l<{\hspost}@{}}%
\column{5}{@{}>{\hspre}c<{\hspost}@{}}%
\column{5E}{@{}l@{}}%
\column{8}{@{}>{\hspre}l<{\hspost}@{}}%
\column{10}{@{}>{\hspre}l<{\hspost}@{}}%
\column{12}{@{}>{\hspre}l<{\hspost}@{}}%
\column{E}{@{}>{\hspre}l<{\hspost}@{}}%
\>[3]{}\Varid{foldMTC}\mathbin{::}\Conid{Algebra}\;\Varid{f}\;\Varid{a}\to \Varid{Fix_{M}}\;\Varid{f}\to \Varid{a}{}\<[E]%
\\
\>[3]{}\Varid{foldMTC}\;\Varid{alg}\;\Varid{fa}\mathrel{=}\Varid{fa}\;\Varid{alg}{}\<[E]%
\\[\blanklineskip]%
\>[3]{}\mathbf{type}\;\Conid{UniversalProperty}\;(\Varid{f}\mathbin{::}\mathbin{*}\to \mathbin{*})\;(\Varid{e}\mathbin{::}\Varid{Fix_{M}}\;\Varid{f}){}\<[E]%
\\
\>[3]{}\hsindent{2}{}\<[5]%
\>[5]{}\mathrel{=}{}\<[5E]%
\>[8]{}\forall \Varid{a}\hsforall \;(\Varid{alg}\mathbin{::}\Conid{Algebra}\;\Varid{f}\;\Varid{a})\;(\Varid{h}\mathbin{::}\Varid{Fix_{M}}\;\Varid{f}\to \Varid{a})\hsdot{\circ }{.}{}\<[E]%
\\
\>[8]{}\hsindent{2}{}\<[10]%
\>[10]{}(\forall \Varid{e}\hsforall \hsdot{\circ }{.}\Varid{h}\;(\Varid{inMTC}\;\Varid{e})\mathrel{=}\Varid{alg}\;\Varid{h}\;\Varid{e})\to {}\<[E]%
\\
\>[10]{}\hsindent{2}{}\<[12]%
\>[12]{}\Varid{h}\;\Varid{e}\mathrel{=}\Varid{foldMTC}\;\Varid{alg}\;\Varid{e}{}\<[E]%
\ColumnHook
\end{hscode}\resethooks
In an initial algebra representation of an inductive datatype, we have
a single implementation of a fold operator that can be proven
correct. In MTC's approach based on Church-encodings however, each
term consists of a separate fold implementation that must satisfy the
universal property.

\end{enumerate}

Hence, in order to enable reasoning MTC must provide a proof of the
universal property of folds for every value of a modular datatype that
is used in a proof. This is mostly done by packaging a term and the
proof of the universal property of its fold in a dependent sum type.
\begin{hscode}\SaveRestoreHook
\column{B}{@{}>{\hspre}l<{\hspost}@{}}%
\column{3}{@{}>{\hspre}l<{\hspost}@{}}%
\column{E}{@{}>{\hspre}l<{\hspost}@{}}%
\>[3]{}\mathbf{type}\;\Conid{FixUP}\;\Varid{f}\mathrel{=}\exists (\Varid{x}\mathbin{::}\Varid{Fix_{M}}\;\Varid{f})\hsexists \hsdot{\circ }{.}\Conid{UniversalProperty}\;\Varid{f}\;\Varid{x}{}\<[E]%
\ColumnHook
\end{hscode}\resethooks
\paragraph{Equality of terms}

Packaging universal properties with terms obfuscates equality of terms
when using Church-encodings. The proof component may differ for the
same underlying term.

This shows up for example in type-soundness proofs in MTC. An
extensible logical relation \ensuremath{\Conid{WTValue}\;(\Varid{v},\Varid{t})} is used to represent
well-typing of values. The judgement ranges over values and types. The
universal properties are needed for inversion lemmas and thus the
judgement needs to range over the variants that are packaged with the
universal properties.

However, knowing that \ensuremath{\Conid{WTValue}\;(\Varid{v},\Varid{t})} and \ensuremath{\Varid{proj1}\;\Varid{t}\mathrel{=}\Varid{proj1}\;\Varid{t'}} does
not directly imply \ensuremath{\Conid{WTValue}\;(\Varid{v},\Varid{t'})} because of the possibly distinct
proof component. To solve this situation auxiliary lemmas are needed
that establish the implication. Other logical relations need similar
lemmas. Every feature that introduces new rules to the judgements must
also provide proof algebras for these lemmas.

In the case study two logical relations need this kind of auxiliary
lemmas: the relation for well-typing and a sub-value relation for
continuity. Both of these relations are indexed by two modular types
and hence need two lemmas each. The proofs of these four lemmas, the
declaration of abstract proof algebras and the use of the lemmas
amounts to roughly 30 LoC per feature.

In our approach we never package proofs together with terms and hence
this problem never appears. We thereby gain better readability of
proofs and a small reduction in code size.

\paragraph{Adequacy}

Adequacy of definitions is an important problem in mechanizations. One
concern is the adequate encoding of fixpoints. MTC relies on a
side-condition to show that for a given functor \ensuremath{\Varid{f}} the folding \ensuremath{\Varid{inMTC}\mathbin{::}\Varid{f}\;(\Varid{Fix_{M}}\;\Varid{f})\to \Varid{Fix_{M}}\;\Varid{f}} and unfolding \ensuremath{\Varid{outMTC}\mathbin{::}\Varid{Fix_{M}}\;\Varid{f}\to \Varid{f}\;(\Varid{Fix_{M}}\;\Varid{f})} are inverse operations, namely, that all appearing \ensuremath{\Varid{Fix_{M}}\;\Varid{f}} values need to have the universal property of folds. This
side-condition raises the question if \ensuremath{\Varid{Fix_{M}}\;\Varid{f}} is an adequate
fixpoint of \ensuremath{\Varid{f}}. The pairing of terms together with their proofs of
the universal property do not form a proper fixpoint either, because
of the possibility of different proof components for the same
underlying terms.

Our approach solve this adequacy issue. The \ensuremath{\Conid{SPF}} type class from
Figure \ref{fig:strictlypositivefunctor} requires that \ensuremath{\mathbf{in}} and \ensuremath{\Varid{out}}
are inverse operations without any side conditions on the values and
containers give rise to proper \ensuremath{\Conid{SPF}} instances.




%% ODER: format ==         = "\mathrel{==}"
%% ODER: format /=         = "\neq "
%
%
\makeatletter
\@ifundefined{lhs2tex.lhs2tex.sty.read}%
  {\@namedef{lhs2tex.lhs2tex.sty.read}{}%
   \newcommand\SkipToFmtEnd{}%
   \newcommand\EndFmtInput{}%
   \long\def\SkipToFmtEnd#1\EndFmtInput{}%
  }\SkipToFmtEnd

\newcommand\ReadOnlyOnce[1]{\@ifundefined{#1}{\@namedef{#1}{}}\SkipToFmtEnd}
\usepackage{amstext}
\usepackage{amssymb}
\usepackage{stmaryrd}
\DeclareFontFamily{OT1}{cmtex}{}
\DeclareFontShape{OT1}{cmtex}{m}{n}
  {<5><6><7><8>cmtex8
   <9>cmtex9
   <10><10.95><12><14.4><17.28><20.74><24.88>cmtex10}{}
\DeclareFontShape{OT1}{cmtex}{m}{it}
  {<-> ssub * cmtt/m/it}{}
\newcommand{\texfamily}{\fontfamily{cmtex}\selectfont}
\DeclareFontShape{OT1}{cmtt}{bx}{n}
  {<5><6><7><8>cmtt8
   <9>cmbtt9
   <10><10.95><12><14.4><17.28><20.74><24.88>cmbtt10}{}
\DeclareFontShape{OT1}{cmtex}{bx}{n}
  {<-> ssub * cmtt/bx/n}{}
\newcommand{\tex}[1]{\text{\texfamily#1}}	% NEU

\newcommand{\Sp}{\hskip.33334em\relax}


\newcommand{\Conid}[1]{\mathit{#1}}
\newcommand{\Varid}[1]{\mathit{#1}}
\newcommand{\anonymous}{\kern0.06em \vbox{\hrule\@width.5em}}
\newcommand{\plus}{\mathbin{+\!\!\!+}}
\newcommand{\bind}{\mathbin{>\!\!\!>\mkern-6.7mu=}}
\newcommand{\rbind}{\mathbin{=\mkern-6.7mu<\!\!\!<}}% suggested by Neil Mitchell
\newcommand{\sequ}{\mathbin{>\!\!\!>}}
\renewcommand{\leq}{\leqslant}
\renewcommand{\geq}{\geqslant}
\usepackage{polytable}

%mathindent has to be defined
\@ifundefined{mathindent}%
  {\newdimen\mathindent\mathindent\leftmargini}%
  {}%

\def\resethooks{%
  \global\let\SaveRestoreHook\empty
  \global\let\ColumnHook\empty}
\newcommand*{\savecolumns}[1][default]%
  {\g@addto@macro\SaveRestoreHook{\savecolumns[#1]}}
\newcommand*{\restorecolumns}[1][default]%
  {\g@addto@macro\SaveRestoreHook{\restorecolumns[#1]}}
\newcommand*{\aligncolumn}[2]%
  {\g@addto@macro\ColumnHook{\column{#1}{#2}}}

\resethooks

\newcommand{\onelinecommentchars}{\quad-{}- }
\newcommand{\commentbeginchars}{\enskip\{-}
\newcommand{\commentendchars}{-\}\enskip}

\newcommand{\visiblecomments}{%
  \let\onelinecomment=\onelinecommentchars
  \let\commentbegin=\commentbeginchars
  \let\commentend=\commentendchars}

\newcommand{\invisiblecomments}{%
  \let\onelinecomment=\empty
  \let\commentbegin=\empty
  \let\commentend=\empty}

\visiblecomments

\newlength{\blanklineskip}
\setlength{\blanklineskip}{0.66084ex}

\newcommand{\hsindent}[1]{\quad}% default is fixed indentation
\let\hspre\empty
\let\hspost\empty
\newcommand{\NB}{\textbf{NB}}
\newcommand{\Todo}[1]{$\langle$\textbf{To do:}~#1$\rangle$}

\EndFmtInput
\makeatother
%
%
%
%
%
%
% This package provides two environments suitable to take the place
% of hscode, called "plainhscode" and "arrayhscode". 
%
% The plain environment surrounds each code block by vertical space,
% and it uses \abovedisplayskip and \belowdisplayskip to get spacing
% similar to formulas. Note that if these dimensions are changed,
% the spacing around displayed math formulas changes as well.
% All code is indented using \leftskip.
%
% Changed 19.08.2004 to reflect changes in colorcode. Should work with
% CodeGroup.sty.
%
\ReadOnlyOnce{polycode.fmt}%
\makeatletter

\newcommand{\hsnewpar}[1]%
  {{\parskip=0pt\parindent=0pt\par\vskip #1\noindent}}

% can be used, for instance, to redefine the code size, by setting the
% command to \small or something alike
\newcommand{\hscodestyle}{}

% The command \sethscode can be used to switch the code formatting
% behaviour by mapping the hscode environment in the subst directive
% to a new LaTeX environment.

\newcommand{\sethscode}[1]%
  {\expandafter\let\expandafter\hscode\csname #1\endcsname
   \expandafter\let\expandafter\endhscode\csname end#1\endcsname}

% "compatibility" mode restores the non-polycode.fmt layout.

\newenvironment{compathscode}%
  {\par\noindent
   \advance\leftskip\mathindent
   \hscodestyle
   \let\\=\@normalcr
   \let\hspre\(\let\hspost\)%
   \pboxed}%
  {\endpboxed\)%
   \par\noindent
   \ignorespacesafterend}

\newcommand{\compaths}{\sethscode{compathscode}}

% "plain" mode is the proposed default.
% It should now work with \centering.
% This required some changes. The old version
% is still available for reference as oldplainhscode.

\newenvironment{plainhscode}%
  {\hsnewpar\abovedisplayskip
   \advance\leftskip\mathindent
   \hscodestyle
   \let\hspre\(\let\hspost\)%
   \pboxed}%
  {\endpboxed%
   \hsnewpar\belowdisplayskip
   \ignorespacesafterend}

\newenvironment{oldplainhscode}%
  {\hsnewpar\abovedisplayskip
   \advance\leftskip\mathindent
   \hscodestyle
   \let\\=\@normalcr
   \(\pboxed}%
  {\endpboxed\)%
   \hsnewpar\belowdisplayskip
   \ignorespacesafterend}

% Here, we make plainhscode the default environment.

\newcommand{\plainhs}{\sethscode{plainhscode}}
\newcommand{\oldplainhs}{\sethscode{oldplainhscode}}
\plainhs

% The arrayhscode is like plain, but makes use of polytable's
% parray environment which disallows page breaks in code blocks.

\newenvironment{arrayhscode}%
  {\hsnewpar\abovedisplayskip
   \advance\leftskip\mathindent
   \hscodestyle
   \let\\=\@normalcr
   \(\parray}%
  {\endparray\)%
   \hsnewpar\belowdisplayskip
   \ignorespacesafterend}

\newcommand{\arrayhs}{\sethscode{arrayhscode}}

% The mathhscode environment also makes use of polytable's parray 
% environment. It is supposed to be used only inside math mode 
% (I used it to typeset the type rules in my thesis).

\newenvironment{mathhscode}%
  {\parray}{\endparray}

\newcommand{\mathhs}{\sethscode{mathhscode}}

% texths is similar to mathhs, but works in text mode.

\newenvironment{texthscode}%
  {\(\parray}{\endparray\)}

\newcommand{\texths}{\sethscode{texthscode}}

% The framed environment places code in a framed box.

\def\codeframewidth{\arrayrulewidth}
\RequirePackage{calc}

\newenvironment{framedhscode}%
  {\parskip=\abovedisplayskip\par\noindent
   \hscodestyle
   \arrayrulewidth=\codeframewidth
   \tabular{@{}|p{\linewidth-2\arraycolsep-2\arrayrulewidth-2pt}|@{}}%
   \hline\framedhslinecorrect\\{-1.5ex}%
   \let\endoflinesave=\\
   \let\\=\@normalcr
   \(\pboxed}%
  {\endpboxed\)%
   \framedhslinecorrect\endoflinesave{.5ex}\hline
   \endtabular
   \parskip=\belowdisplayskip\par\noindent
   \ignorespacesafterend}

\newcommand{\framedhslinecorrect}[2]%
  {#1[#2]}

\newcommand{\framedhs}{\sethscode{framedhscode}}

% The inlinehscode environment is an experimental environment
% that can be used to typeset displayed code inline.

\newenvironment{inlinehscode}%
  {\(\def\column##1##2{}%
   \let\>\undefined\let\<\undefined\let\\\undefined
   \newcommand\>[1][]{}\newcommand\<[1][]{}\newcommand\\[1][]{}%
   \def\fromto##1##2##3{##3}%
   \def\nextline{}}{\) }%

\newcommand{\inlinehs}{\sethscode{inlinehscode}}

% The joincode environment is a separate environment that
% can be used to surround and thereby connect multiple code
% blocks.

\newenvironment{joincode}%
  {\let\orighscode=\hscode
   \let\origendhscode=\endhscode
   \def\endhscode{\def\hscode{\endgroup\def\@currenvir{hscode}\\}\begingroup}
   %\let\SaveRestoreHook=\empty
   %\let\ColumnHook=\empty
   %\let\resethooks=\empty
   \orighscode\def\hscode{\endgroup\def\@currenvir{hscode}}}%
  {\origendhscode
   \global\let\hscode=\orighscode
   \global\let\endhscode=\origendhscode}%

\makeatother
\EndFmtInput
%
%
%
% First, let's redefine the forall, and the dot.
%
%
% This is made in such a way that after a forall, the next
% dot will be printed as a period, otherwise the formatting
% of `comp_` is used. By redefining `comp_`, as suitable
% composition operator can be chosen. Similarly, period_
% is used for the period.
%
\ReadOnlyOnce{forall.fmt}%
\makeatletter

% The HaskellResetHook is a list to which things can
% be added that reset the Haskell state to the beginning.
% This is to recover from states where the hacked intelligence
% is not sufficient.

\let\HaskellResetHook\empty
\newcommand*{\AtHaskellReset}[1]{%
  \g@addto@macro\HaskellResetHook{#1}}
\newcommand*{\HaskellReset}{\HaskellResetHook}

\global\let\hsforallread\empty

\newcommand\hsforall{\global\let\hsdot=\hsperiodonce}
\newcommand*\hsperiodonce[2]{#2\global\let\hsdot=\hscompose}
\newcommand*\hscompose[2]{#1}

\AtHaskellReset{\global\let\hsdot=\hscompose}

% In the beginning, we should reset Haskell once.
\HaskellReset

\makeatother
\EndFmtInput

% format runC         = "run\mathbb{C}_{\scalebox{0.6}{T}}"
% format C            = "\mathbb{C}_{\scalebox{0.6}{T}}"
%%format R            = "\mathbb{R}_{\scalebox{0.6}{T}}"


\section{Related and Future Work}

\paragraph{DGP in proof-assistants}

Datatype-generic programming started out as a form of language
extension such as PolyP \cite{jansson:polyp} or Generic Haskell
\cite{loh:dsgh}. Yet Haskell has turned out to be powerful enough to
implement datatype-generic programming in the language itself and over
the time a vast number of DGP libraries for Haskell have been proposed
\cite{cheney:ligd,syb,emgm,multirec,instantgenerics,uniplate,genericderiving}. Compared
with a language extension, a library is much easier to develop and
maintain.

Using the flexibility of dependent-types there are multiple proposals
for performing datatype-generic programming in proof assistants
\cite{dgpcoq,altenkirch:gpwdtp,benke:universes,loh:gpif,indexedcontainers}. This
setting not only allows the implementation of generic functions, but
also of generic proofs. The approaches vary in terms of how strictly
they follow the positivity or termination restrictions imposed by the
proof assistant. Some circumvent the type-checker at various points to
simplify the development or presentation while others put more effort
in convincing the type-checker and termination checker of the validity
\cite{ertt}. However, in all of the proposals there does not seem to
be any fundamental problem caused by the restrictions.


\paragraph{DGP for modular proofs}

Modularly composing semantics and proofs about the semantics has also
been addressed by \cite{schwaab:mtp} in the context of programming
language meta-theory. They perform their development in Agda and
similar to our approach they also use a universe approach based on
polynomial functors to represent modular datatypes. They split
relations for small-step operational semantics and well-typing on a
feature basis. However, the final fixed points are constructed manually
instead of having a generic representation of inductive families.

Using Coq's type classes both MTC and our approach also allow for more
automation in the final composition of datatypes, functions and
proofs. Agda features instance arguments that can be used to replace
type classes in various cases. However, the current implementation
does not perform recursive resolution and as a result Agda does not
support automation of the composition to the extent that is needed for
DTC-like approaches.


\paragraph{Combining different DGP approaches}

We have shown an embedding of the universe of polynomial functors into
the universe of containers. Similar inclusions between universes have
been done in the literature \cite{morris:cspf}. Magalh\~aes and L\"oh
\cite{jpm:fcadgp} have ported several popular DGP approaches from
Haskell to Agda and performed a formal comparison by proving inclusion
relations between the approaches.

DGP approaches differ in terms of the class of datatypes they capture and the
set of generic functions that can be implemented for them. Generic functions
can be transported from a universe into a sub-universe. However, we are not
aware of any DGP library with a systematic treatment of universes where each generic
function is defined on the most general universe that supports that function.


\paragraph{DGP for abstract syntax}

We have shown how to obtain more reuse by complementing modular
definitions with a generic equality function and generic proofs of its
properties. Of course more generic functionality like traversals,
pretty-printing, parsing etc. can be covered by means of
datatype-generic programming.

One very interesting idea is to use datatype-generic programming to
handle variable binding \cite{cheney:synp,unbound}. Variable
binding is an ubiquitous aspect of programming languages. Moreover, a
lot of functionality like variable substitutions and free variable
calculations is defined for many languages. Licata and Harper
\cite{licata:ubc} and Keuchel and Jeuring \cite{sk:gcasr} define
universes for datatypes with binding in Agda. Lee et al.~\cite{gmeta}
develop a framework for first-order representations of variable
binding in Coq that is based on the universe of regular tree types
\cite{ertt} and that provides many of the so-called
\emph{infrastructure lemmas} required when mechanizing programming
language meta-theory.

An interesting direction for future work is to extend these approaches
to capture variable binding in the indices of relations on abstract
syntax and use this as the underlying representation of extensible
datatypes and extensible logical relations and thereby complementing
modular functions with generic proofs about variable binding.


\paragraph{Automatic derivation of instances}

Most, if not all, generic programming libraries in Haskell use Template
Haskell to derive the generic representation of user-defined types and
to derive the conversion functions between them.

The GMeta \cite{gmeta} framework includes a standalone tool that also
performs this derivation for Coq. Similarly we also like to be able
to derive instances for the \ensuremath{\Conid{Container}} and \ensuremath{\Conid{Polynomial}} classes
automatically.

\chapter{Conclusion}

In this concluding chapter we revisit the research question, summarize the
results of this thesis and evaluate the contributions in the context of the
research question. Furthermore, we highlight opportunities for future work and
speculate how the contributions of this thesis can benefit the programming
language research community at large in the future.

\section{Research Question}
Widespread formalization and mechanisation of programming language meta-theory
is hindered by the large development costs. For further adoption, reducing the
costs is crucial. This leads us back to the research question laid out in the
beginning of this thesis:

\begin{center}
  \begin{minipage}{0.8\columnwidth}\bf
    How can we reduce the cost of mechanising the formal meta-theory of
    programming languages?
  \end{minipage}
\end{center}

This thesis promotes reuse as a way to reduce the mechanisation effort of
programming language meta-theory and investigates the two principled approaches
\emph{modularity} and \emph{genericity} to reuse in general, and the application
of these approaches to programming language theory in particular.

\section{Summary}

This section gives and overview of the thesis and its most important results. We
discuss the two pats of the thesis in turn.

\subsection{Modularity}
Part I of this thesis pursues the modularity approach and universal method for
modularization of functional programs on inductive datatypes and modularization
of induction proofs for properties of these programs. Specifically this thesis
contributed new results for modular representation of modular datatypes and on
reducing feature interaction for effectful semantics. Both contributions have
been evaluated using case studies. We discuss both contributions and the case
studies below.

%% This universal method is showcased by modularizing type-safety proofs for lambda
%% calculi with several different features.

\subsubsection{Modular Representation}
Modularizing proofs is particularly challenging since the proof-assistant
settings imposes several restrictions for logical consistency. This challenge is
addressed in work prior to this thesis in the \emph{Meta-Theory \`a la Carte
  (MTC)} \cite{mtc} framework which uses Church encodings to represent inductive
datatypes and families. This thesis develops and alternative approach using
datatype-generic representations that improves over MTC in adequacy, convenience
and compatibility with classical logic.

\subsubsection{Feature Interaction}
A concern in modularization is the interaction between two or more features.
Each new feature that is integrated potentially interacts with all previous
ones. As a result, extending existing developments exhibits a quadratic increase
in effort. This is not a problem that is specific to a modular approach, but
applies to software development in general. However, it is an obstacle to
modularity if the interaction involves the complete set of features instead of
an isolated subset, e.g. two features.

Problematic feature interactions appear in the semantics of programming
languages with effectful features. To cut down this particular kind of
interaction, we developed a denotational semantics based on monad transformers
and corresponding algebraic laws. This semantics allowed us to to formulate and
prove a general type-soundness theorem in a modular way that enables the modular
reuse of language feature proofs and reduce the interaction between effectful
language features to the interaction between their effects.

\subsubsection{Case Studies}

We performed two case studies to evaluate our contributions. The first case
study is a port of the MTC case study to the container based modular
representation. It consists of continuity and type-safety proofs for seven
language features and six language compositions. In contrast to MTC, the
preservation of the universal properties or any other property is not necessary
for induction which makes the approach conceptually simpler. This is reflected
in the case study: sigma types and its projections are removed from lemmas and
proofs, and language features and their (proof) algebras are on average 240 LoC
(25\%) smaller. The final language compositions are slightly larger by about 6
LoC (7\%), but are short in comparison to the feature specific code.

The second case study demonstrates the 3MT framework. It consists of five
reusable language features: pure boolean and arithmetic features and effectful
features for references, errors and lambda abstractions. The study builds twenty
eight different combinations of the features. Each language feature has its own
reusable feature theorem and each of the six effect combinations its own effect
theorem. Effect theorems are only reusable for languages with the specific set
of effects. Like in the first case study, the feature and effect theorems
outweigh the final language compositions.



\subsection{Genericity}
Part II examines the genericity approach with a specific use case in mind:
variable binding boilerplate in mechanisations. A key ingredient of
reduction-based semantics of programming languages is substitiution.
Meta-theoretic proofs using these semantics usually involves a large number of
burdensome boilerplate proofs about substitutions which distracts a human
semanticist from essential theorems when mechanising her proofs. This thesis
develops a generic solution to the boilerplate lemmas based on a novel
specification language \Knot for programming language and a code generator
\Needle that produces boilerplate code. We summarize and discuss the
contributions.

\subsubsection{Specification Language}
\Knot allows the specification of abstract syntax, with explicit specifications
of binding, and of semantical relations between syntax terms. Relations are
defined using arbitrary expressions build from a language's abstract syntax and
a type system is used to check that all expressions are well-scoped.

\paragraph{Free Monadic Structure}
A central contribution of this thesis is the identification of a large class of
syntaxes for which boilerplate is completely generically derivable: syntactic
sorts that have a free monad-like structure. For relations this translates to
context parametricity of regular (non-variable) rules.

\paragraph{Principled Elaboration}
\Needle produces specialized definitions for a given \Knot specification in the
\Coq proof assistant and produces code for boilerplate functions and boilerplate
lemmas. We employ a principled approach to elaboration of boilerplate code that
gives us confidence in the correctness of our code generator: we developed and
formally verified elaborations in the \Coq proof assistant using our
datatype-generic implementation \Loom of \Knot.

\paragraph{Case Study}
Our case study compares our generic approach against fully manual mechanisations
of type safety proofs of 10 lambda calculi. It shows substantial savings in the
mechanisation for all 10 calculi with the largest savings being about 74\%
reduction in code size for System~F. This case study indicates that replacing
manual variable binding boilerplate by reusable generic solutions is indeed an
effective means of reducing the mechanisation effort.


\section{Future Work}

This section presents possible directions for further research. We group the
ideas around three themes: convenience, scale and mathematical foundation.

\subsection{Convenience}

An unanswered question is how well modularity fared in light of our research
question. We achieved our intermediate goal of developing an approach to modular
and reusable components for language features and their type-safety proofs and
reduction of non-modular feature interaction between effectful features.

However, the approach requires substantial bookkeeping of the relationship of
final and intermediate compositions of datatypes, relations, algebras and proof
algebras and similarly between the final and intermediate compositions of
effects for our monadic denotational semantics. This bookkeeping is not only
inconvenient but also puts additional burden on mechanizers which increases the
mechanization effort. Furthermore, the shift to modular algebras and proof
algebras may be unfamiliar so that the approach seems inconvenient over
traditional monolithic recursive functions and induction proofs.

For any practical use we need better or even direct support from the proof
assistant. Concepts like \emph{proof algebras} presented in this thesis can be
turned into first-class primitives of the proof-assistant's language. The
generic universe implementation of the first part can be used as a basis for an
elaboration into a core calculus.

\begin{itemize}
\item Scale \Needle \& \Knot too.
\end{itemize}

\subsection{Scale}

There are multiple possibilities to grow the \Knot specification language to
handle more object languages. For instance, we can extend \Knot with support for
programming and let the user write functions. These functions could then also be
used in the definition of relations by extending the expression language
accordingly. Additionally, we can import more concepts like GADTs and derive
boilerplate for intrinsically well-typed syntax and include dependent types in
the expression language.

Furthermore, we can integrate more variable binding like concepts into \Knot,
for instance, first-class substitutions. This will also require us to improve
\Knot scope-checking type system.

The \Needle the code generator can be scaled to include the above extensions.
However, \Needle can also be scaled independently. Currently, the code is geared
towards type safety proofs, but other meta-theoretic proofs may require
different boilerplate that could be generated by \Needle. Furthermore, \Knot
itself is independent of a particular representation. Hence, we can imagine
generating boilerplate for different representations like a nominal,
locally-named or locally-nameless representation.


\subsection{Mathematical Foundations}

Last but not least, it would be interesting to explore the mathematical
foundations that underly \Knot. Categorical models exist for syntax with
variable binding, but \Knot's features exceed what can be described in these
models. For instance, the usual limitation is that only one namespace is
assumed.

A promising area is the modelling of well-scoped terms as a generalization of
relative monads. This is all the more interesting, since there is already
research available that models semantic relations as modules over relative monads.



%%% Local Variables:
%%% mode: latex
%%% TeX-master: "../main"
%%% End:
