
\section{Metatheory}
Meta-theory of programming languages is a core topic of computer science that
concerns itself with the formalization of propositions about programming
languages, their semantics and related systems like type systems.

The meta-theory of programming language semantics and type-systems is highly
complex due to the management of many details. Formal proofs are long and prone
to subtle errors that can invalidate large amounts of work. In order to
guarantee the correctness of formal meta-theory, techniques for mechanical
formalization in proof-assistants have received much attention in recent years.


To lighten the burden of programming language mechanization, many approaches
have been developed that tackle the substantial boilerplate which arises from
variable binders. Unfortunately, existing approaches for first-order
representations are limited to the boilerplate that concerns the syntax of
languages and do not tackle common boilerplate lemmas that arise for semantic
relations such as typing. Consequently, the human mechanizer is still burdened
with proving these substantial boilerplate lemmas manually.
