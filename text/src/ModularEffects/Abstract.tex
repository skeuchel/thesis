This paper presents \Name, a framework for \emph{modular}
mechanized meta-theory of languages with \emph{effects}. Using \Name,
individual language features and their corresponding definitions --
\emph{semantic functions}, \emph{theorem statements} and \emph{proofs}
-- can be built separately and then reused to create different
languages with fully mechanized meta-theory.  \name combines
\emph{modular datatypes} and \emph{monads} to define
denotational semantics with effects on a per-feature basis, without fixing the
particular set of effects or language constructs.

% Existing work already provides three important ingredients towards this goal:
% \emph{monads}, \emph{modular datatypes} and \emph{modular induction}. With
% monads and modular datatypes semantic functions can be modularly defined for
% each language feature without hard-wiring a particular set of effects or
% language constructs. 

One well-established problem with \emph{type soundness} proofs for
denotational semantics is that they are notoriously brittle with
respect to the addition of new effects.
%To modularize proofs about these effectful semantic functions, \name shows that
%it is crucial to formulate the right theorems. 
The statement of type soundness for a language depends intimately on the
effects it uses, making it particularly
challenging to achieve modularity. \name solves this long-standing problem by splitting these theorems into
two separate and reusable parts: a \emph{feature theorem} that captures
the well-typing of denotations
produced by the semantic function of an invidual feature with respect
to only the effects used,
and an \emph{effect theorem} that adapts well-typings of denotations to a
fixed superset of effects.
The proof of type soundness for a particular language simply combines these theorems for its features
and the combination of their effects.
%
To establish both theorems, \name uses two key reasoning techniques:
\emph{modular induction} and \emph{algebraic laws} about effects. Several
effectful language features, including references and errors, illustrate the
capabilities of \Name. A case study reuses these features to build fully
mechanized definitions and proofs for 28 languages, including several
versions of mini-ML with effects.


% The main contribution of this paper is to show how to modularize the theorems
% and proofs for these modular monadic definitions. 

\begin{comment}
Monads are needed to avoid 
hard-wiring a fixed set of effects into the definitions; 
modular datatypes avoid hard-wiring a fixed set of language 
constructs; and modular induction 

Previous work also shows that the
combination of \emph{monads} and \emph{modular datatypes} 
is sufficient to effectively modularize semantic functions. 
However for modularizing theorem statements (such as 
\emph{type-soundness}) and corresponding proofs we also 
need modular induction. 

All three ingredients are needed to
effectively modularize theorem statements and proofs. 

The combination of monads and modular
datatypes has already been shown to effectively deal with the
modularization of semantics functions. 



Monads are a well-established 
providing the semantics of languages with effects; with the help
of qualified types and monad transformers, monads can be modularly
composed. 

 \emph{Monads} and \emph{monad transformers}
are a well-established way 

we have already shown how to modularize languages without
effects. When adding effects every definition -- \emph{semantic
functions}, \emph{theorem statements} and \emph{proofs} -- has
to be suitable generalized to accounts for potential effects. 
For semantic functions, \emph{monads} and \emph{monad transformers}
have already been shown as an effective way to generalize 
such definitions.

  

The main challange in adding effects to the modular
meta-theory framework is how to generalize \emph{theorem statements}
(such as \emph{type-soundness}) and corresponding \emph{proofs} to a
form that is \emph{general} enough to account for all potential 
effects 

Hence, the key question is what form the general theorem should take. It cannot hardwire the effects, but must instead specif- ically cater to establishing type soundness with respect to any de- sired set of effects. 


This
framework builds on previous work which has already shown how to
modularize meta-theory using modular encodings of datatypes with a
corresponding (modular) induction principle. However that work
was limited to languages without effects. The challange in adding
effects is they essentially change every definition:
\emph{semantic functions}, \emph{theorem statements} and
\emph{proofs}. Fortunately, for semantic functions there is already a
good solution: \emph{monads}.  Monads are a well-established mechanism
for providing the semantics of languages with effects; with the help
of qualified types and monad transformers, monads can be modularly
composed. However, dealing with the corresponding modular monadic theorem
statements and proofs offers significant new challenges that have not
been addressed before. In particular defining a modular
\emph{type-soundness} statement is non-trivial because existing
non-modular type-soundness theorems typically assume a concrete set of
effects that is required for the particular proof in hand.  Since we
cannot assume the concrete set of effects, We overcome this problem by
generilize the theorem statement and prove type-soundness in two
steps.

Dealing modularly with effects
is challenging because effects essentially change every definition:
\emph{semantic functions}, \emph{theorem statements} and
\emph{proofs}.  Fortunately, for semantic functions there is already a
good solution: \emph{monads}.  Monads are a well-established mechanism
for providing the semantics of languages with effects; with the help
of qualified types and monad transformers, monads can be modularly
composed.


When dealing with languages with
effects the set of effects for different, modularly defined language
features, may be different. As a result it is not possible to
antecipate the concrete 

In previous work we have shown how to modularize
meta-theory using Church encodings of datatypes with a corresponding
(extensible) induction principle. An important limitation of
that work was that it could only handle \emph{pure} languages.





However, dealing with
corresponding modular monadic theorem statements and proofs offers
significant new challenges that have not been addressed before. In
particular defining a modular \emph{type-soundness} statement is
challanging because existing non-modular type-soundness theorems
typically assume a concrete set of effects that is required for the
particular proof in hand.  Since we cannot assume the concrete set of
effects, We overcome this problem by generilize the theorem statement
and prove type-soundness in two steps.

we use algebraic laws on monads at a fundamental level.
\end{comment}
