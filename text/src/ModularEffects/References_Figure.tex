%% ODER: format ==         = "\mathrel{==}"
%% ODER: format /=         = "\neq "
%
%
\makeatletter
\@ifundefined{lhs2tex.lhs2tex.sty.read}%
  {\@namedef{lhs2tex.lhs2tex.sty.read}{}%
   \newcommand\SkipToFmtEnd{}%
   \newcommand\EndFmtInput{}%
   \long\def\SkipToFmtEnd#1\EndFmtInput{}%
  }\SkipToFmtEnd

\newcommand\ReadOnlyOnce[1]{\@ifundefined{#1}{\@namedef{#1}{}}\SkipToFmtEnd}
\usepackage{amstext}
\usepackage{amssymb}
\usepackage{stmaryrd}
\DeclareFontFamily{OT1}{cmtex}{}
\DeclareFontShape{OT1}{cmtex}{m}{n}
  {<5><6><7><8>cmtex8
   <9>cmtex9
   <10><10.95><12><14.4><17.28><20.74><24.88>cmtex10}{}
\DeclareFontShape{OT1}{cmtex}{m}{it}
  {<-> ssub * cmtt/m/it}{}
\newcommand{\texfamily}{\fontfamily{cmtex}\selectfont}
\DeclareFontShape{OT1}{cmtt}{bx}{n}
  {<5><6><7><8>cmtt8
   <9>cmbtt9
   <10><10.95><12><14.4><17.28><20.74><24.88>cmbtt10}{}
\DeclareFontShape{OT1}{cmtex}{bx}{n}
  {<-> ssub * cmtt/bx/n}{}
\newcommand{\tex}[1]{\text{\texfamily#1}}	% NEU

\newcommand{\Sp}{\hskip.33334em\relax}


\newcommand{\Conid}[1]{\mathit{#1}}
\newcommand{\Varid}[1]{\mathit{#1}}
\newcommand{\anonymous}{\kern0.06em \vbox{\hrule\@width.5em}}
\newcommand{\plus}{\mathbin{+\!\!\!+}}
\newcommand{\bind}{\mathbin{>\!\!\!>\mkern-6.7mu=}}
\newcommand{\rbind}{\mathbin{=\mkern-6.7mu<\!\!\!<}}% suggested by Neil Mitchell
\newcommand{\sequ}{\mathbin{>\!\!\!>}}
\renewcommand{\leq}{\leqslant}
\renewcommand{\geq}{\geqslant}
\usepackage{polytable}

%mathindent has to be defined
\@ifundefined{mathindent}%
  {\newdimen\mathindent\mathindent\leftmargini}%
  {}%

\def\resethooks{%
  \global\let\SaveRestoreHook\empty
  \global\let\ColumnHook\empty}
\newcommand*{\savecolumns}[1][default]%
  {\g@addto@macro\SaveRestoreHook{\savecolumns[#1]}}
\newcommand*{\restorecolumns}[1][default]%
  {\g@addto@macro\SaveRestoreHook{\restorecolumns[#1]}}
\newcommand*{\aligncolumn}[2]%
  {\g@addto@macro\ColumnHook{\column{#1}{#2}}}

\resethooks

\newcommand{\onelinecommentchars}{\quad-{}- }
\newcommand{\commentbeginchars}{\enskip\{-}
\newcommand{\commentendchars}{-\}\enskip}

\newcommand{\visiblecomments}{%
  \let\onelinecomment=\onelinecommentchars
  \let\commentbegin=\commentbeginchars
  \let\commentend=\commentendchars}

\newcommand{\invisiblecomments}{%
  \let\onelinecomment=\empty
  \let\commentbegin=\empty
  \let\commentend=\empty}

\visiblecomments

\newlength{\blanklineskip}
\setlength{\blanklineskip}{0.66084ex}

\newcommand{\hsindent}[1]{\quad}% default is fixed indentation
\let\hspre\empty
\let\hspost\empty
\newcommand{\NB}{\textbf{NB}}
\newcommand{\Todo}[1]{$\langle$\textbf{To do:}~#1$\rangle$}

\EndFmtInput
\makeatother
%
%
%
%
%
%
% This package provides two environments suitable to take the place
% of hscode, called "plainhscode" and "arrayhscode". 
%
% The plain environment surrounds each code block by vertical space,
% and it uses \abovedisplayskip and \belowdisplayskip to get spacing
% similar to formulas. Note that if these dimensions are changed,
% the spacing around displayed math formulas changes as well.
% All code is indented using \leftskip.
%
% Changed 19.08.2004 to reflect changes in colorcode. Should work with
% CodeGroup.sty.
%
\ReadOnlyOnce{polycode.fmt}%
\makeatletter

\newcommand{\hsnewpar}[1]%
  {{\parskip=0pt\parindent=0pt\par\vskip #1\noindent}}

% can be used, for instance, to redefine the code size, by setting the
% command to \small or something alike
\newcommand{\hscodestyle}{}

% The command \sethscode can be used to switch the code formatting
% behaviour by mapping the hscode environment in the subst directive
% to a new LaTeX environment.

\newcommand{\sethscode}[1]%
  {\expandafter\let\expandafter\hscode\csname #1\endcsname
   \expandafter\let\expandafter\endhscode\csname end#1\endcsname}

% "compatibility" mode restores the non-polycode.fmt layout.

\newenvironment{compathscode}%
  {\par\noindent
   \advance\leftskip\mathindent
   \hscodestyle
   \let\\=\@normalcr
   \let\hspre\(\let\hspost\)%
   \pboxed}%
  {\endpboxed\)%
   \par\noindent
   \ignorespacesafterend}

\newcommand{\compaths}{\sethscode{compathscode}}

% "plain" mode is the proposed default.
% It should now work with \centering.
% This required some changes. The old version
% is still available for reference as oldplainhscode.

\newenvironment{plainhscode}%
  {\hsnewpar\abovedisplayskip
   \advance\leftskip\mathindent
   \hscodestyle
   \let\hspre\(\let\hspost\)%
   \pboxed}%
  {\endpboxed%
   \hsnewpar\belowdisplayskip
   \ignorespacesafterend}

\newenvironment{oldplainhscode}%
  {\hsnewpar\abovedisplayskip
   \advance\leftskip\mathindent
   \hscodestyle
   \let\\=\@normalcr
   \(\pboxed}%
  {\endpboxed\)%
   \hsnewpar\belowdisplayskip
   \ignorespacesafterend}

% Here, we make plainhscode the default environment.

\newcommand{\plainhs}{\sethscode{plainhscode}}
\newcommand{\oldplainhs}{\sethscode{oldplainhscode}}
\plainhs

% The arrayhscode is like plain, but makes use of polytable's
% parray environment which disallows page breaks in code blocks.

\newenvironment{arrayhscode}%
  {\hsnewpar\abovedisplayskip
   \advance\leftskip\mathindent
   \hscodestyle
   \let\\=\@normalcr
   \(\parray}%
  {\endparray\)%
   \hsnewpar\belowdisplayskip
   \ignorespacesafterend}

\newcommand{\arrayhs}{\sethscode{arrayhscode}}

% The mathhscode environment also makes use of polytable's parray 
% environment. It is supposed to be used only inside math mode 
% (I used it to typeset the type rules in my thesis).

\newenvironment{mathhscode}%
  {\parray}{\endparray}

\newcommand{\mathhs}{\sethscode{mathhscode}}

% texths is similar to mathhs, but works in text mode.

\newenvironment{texthscode}%
  {\(\parray}{\endparray\)}

\newcommand{\texths}{\sethscode{texthscode}}

% The framed environment places code in a framed box.

\def\codeframewidth{\arrayrulewidth}
\RequirePackage{calc}

\newenvironment{framedhscode}%
  {\parskip=\abovedisplayskip\par\noindent
   \hscodestyle
   \arrayrulewidth=\codeframewidth
   \tabular{@{}|p{\linewidth-2\arraycolsep-2\arrayrulewidth-2pt}|@{}}%
   \hline\framedhslinecorrect\\{-1.5ex}%
   \let\endoflinesave=\\
   \let\\=\@normalcr
   \(\pboxed}%
  {\endpboxed\)%
   \framedhslinecorrect\endoflinesave{.5ex}\hline
   \endtabular
   \parskip=\belowdisplayskip\par\noindent
   \ignorespacesafterend}

\newcommand{\framedhslinecorrect}[2]%
  {#1[#2]}

\newcommand{\framedhs}{\sethscode{framedhscode}}

% The inlinehscode environment is an experimental environment
% that can be used to typeset displayed code inline.

\newenvironment{inlinehscode}%
  {\(\def\column##1##2{}%
   \let\>\undefined\let\<\undefined\let\\\undefined
   \newcommand\>[1][]{}\newcommand\<[1][]{}\newcommand\\[1][]{}%
   \def\fromto##1##2##3{##3}%
   \def\nextline{}}{\) }%

\newcommand{\inlinehs}{\sethscode{inlinehscode}}

% The joincode environment is a separate environment that
% can be used to surround and thereby connect multiple code
% blocks.

\newenvironment{joincode}%
  {\let\orighscode=\hscode
   \let\origendhscode=\endhscode
   \def\endhscode{\def\hscode{\endgroup\def\@currenvir{hscode}\\}\begingroup}
   %\let\SaveRestoreHook=\empty
   %\let\ColumnHook=\empty
   %\let\resethooks=\empty
   \orighscode\def\hscode{\endgroup\def\@currenvir{hscode}}}%
  {\origendhscode
   \global\let\hscode=\orighscode
   \global\let\endhscode=\origendhscode}%

\makeatother
\EndFmtInput
%
%
%
% First, let's redefine the forall, and the dot.
%
%
% This is made in such a way that after a forall, the next
% dot will be printed as a period, otherwise the formatting
% of `comp_` is used. By redefining `comp_`, as suitable
% composition operator can be chosen. Similarly, period_
% is used for the period.
%
\ReadOnlyOnce{forall.fmt}%
\makeatletter

% The HaskellResetHook is a list to which things can
% be added that reset the Haskell state to the beginning.
% This is to recover from states where the hacked intelligence
% is not sufficient.

\let\HaskellResetHook\empty
\newcommand*{\AtHaskellReset}[1]{%
  \g@addto@macro\HaskellResetHook{#1}}
\newcommand*{\HaskellReset}{\HaskellResetHook}

\global\let\hsforallread\empty

\newcommand\hsforall{\global\let\hsdot=\hsperiodonce}
\newcommand*\hsperiodonce[2]{#2\global\let\hsdot=\hscompose}
\newcommand*\hscompose[2]{#1}

\AtHaskellReset{\global\let\hsdot=\hscompose}

% In the beginning, we should reset Haskell once.
\HaskellReset

\makeatother
\EndFmtInput

% format runC         = "run\mathbb{C}_{\scalebox{0.6}{T}}"
% format C            = "\mathbb{C}_{\scalebox{0.6}{T}}"
%%format R            = "\mathbb{R}_{\scalebox{0.6}{T}}"

















%%format ._ = "."









\begin{figure*}[!ht]

\scriptsize
\begin{center}
\begin{tabular}{|ccc|}
\hline
\hspace{-.5cm}
\begin{minipage}[t]{0.23\linewidth}
\vspace{-.15cm}
\hspace{-5pt}

\hspace{.65cm}\ruleline{\scriptsize Simplified value interface}\begin{hscode}\SaveRestoreHook
\column{B}{@{}>{\hspre}l<{\hspost}@{}}%
\column{3}{@{}>{\hspre}l<{\hspost}@{}}%
\column{10}{@{}>{\hspre}l<{\hspost}@{}}%
\column{E}{@{}>{\hspre}l<{\hspost}@{}}%
\>[3]{}\mathbf{type}\;\Conid{Value}{}\<[E]%
\\[\blanklineskip]%
\>[3]{}\Varid{loc}{}\<[10]%
\>[10]{}\mathbin{::}\Conid{Int}\to \Conid{Value}{}\<[E]%
\\
\>[3]{}\Varid{stuck}{}\<[10]%
\>[10]{}\mathbin{::}\Conid{Value}{}\<[E]%
\\
\>[3]{}\Varid{unit}{}\<[10]%
\>[10]{}\mathbin{::}\Conid{Value}{}\<[E]%
\\
\>[3]{}\Varid{isLoc}{}\<[10]%
\>[10]{}\mathbin{::}\Conid{Value}\to \Conid{Maybe}\;\Conid{Int}{}\<[E]%
\ColumnHook
\end{hscode}\resethooks
\vspace{-3pt}

\hspace{.65cm}\ruleline{\scriptsize Simplified type interface}\begin{hscode}\SaveRestoreHook
\column{B}{@{}>{\hspre}l<{\hspost}@{}}%
\column{3}{@{}>{\hspre}l<{\hspost}@{}}%
\column{11}{@{}>{\hspre}l<{\hspost}@{}}%
\column{E}{@{}>{\hspre}l<{\hspost}@{}}%
\>[3]{}\mathbf{type}\;\Conid{Type}{}\<[E]%
\\[\blanklineskip]%
\>[3]{}\Varid{tRef}{}\<[11]%
\>[11]{}\mathbin{::}\Conid{Type}\to \Conid{Type}{}\<[E]%
\\
\>[3]{}\Varid{tUnit}{}\<[11]%
\>[11]{}\mathbin{::}\Conid{Type}{}\<[E]%
\\
\>[3]{}\Varid{isTRef}{}\<[11]%
\>[11]{}\mathbin{::}\Conid{Type}\to \Conid{Maybe}\;\Conid{Type}{}\<[E]%
\ColumnHook
\end{hscode}\resethooks
\vspace{-3pt}

\hspace{.65cm}\ruleline{\scriptsize Expression functor}\begin{hscode}\SaveRestoreHook
\column{B}{@{}>{\hspre}l<{\hspost}@{}}%
\column{3}{@{}>{\hspre}l<{\hspost}@{}}%
\column{15}{@{}>{\hspre}l<{\hspost}@{}}%
\column{E}{@{}>{\hspre}l<{\hspost}@{}}%
\>[3]{}\mathbf{data}\;\Varid{Ref}_F\;\Varid{a}\mathrel{=}\Conid{Ref}\;\Varid{a}{}\<[E]%
\\
\>[3]{}\hsindent{12}{}\<[15]%
\>[15]{}\mid \Conid{DeRef}\;\Varid{a}{}\<[E]%
\\
\>[3]{}\hsindent{12}{}\<[15]%
\>[15]{}\mid \Conid{Assign}\;\Varid{a}\;\Varid{a}{}\<[E]%
\ColumnHook
\end{hscode}\resethooks
\begin{hscode}\SaveRestoreHook
\column{B}{@{}>{\hspre}l<{\hspost}@{}}%
\column{3}{@{}>{\hspre}l<{\hspost}@{}}%
\column{E}{@{}>{\hspre}l<{\hspost}@{}}%
\>[3]{}\mathbf{type}\;\Conid{Store}\mathrel{=}[\mskip1.5mu \Conid{Value}\mskip1.5mu]{}\<[E]%
\ColumnHook
\end{hscode}\resethooks
\end{minipage} & 

\begin{minipage}[t]{0.28\linewidth}
\vspace{-.05cm}

\hspace{.65cm}\ruleline{\scriptsize Monadic typing algebra}\begin{hscode}\SaveRestoreHook
\column{B}{@{}>{\hspre}l<{\hspost}@{}}%
\column{3}{@{}>{\hspre}l<{\hspost}@{}}%
\column{5}{@{}>{\hspre}l<{\hspost}@{}}%
\column{9}{@{}>{\hspre}l<{\hspost}@{}}%
\column{12}{@{}>{\hspre}l<{\hspost}@{}}%
\column{13}{@{}>{\hspre}l<{\hspost}@{}}%
\column{26}{@{}>{\hspre}l<{\hspost}@{}}%
\column{29}{@{}>{\hspre}l<{\hspost}@{}}%
\column{E}{@{}>{\hspre}l<{\hspost}@{}}%
\>[3]{}\Varid{typeof_{Ref}}\mathbin{::}\mathbb{F}_{\scalebox{0.6}{M}}\;\Varid{m}\Rightarrow {}\<[E]%
\\
\>[3]{}\hsindent{10}{}\<[13]%
\>[13]{}\Varid{Algebra}_M\;\Varid{Ref}_F\;(\Varid{m}\;\Conid{Type}){}\<[E]%
\\
\>[3]{}\Varid{typeof_{Ref}}\;\Varid{rec}\;(\Conid{Ref}\;\Varid{e})\mathrel{=}{}\<[E]%
\\
\>[3]{}\hsindent{2}{}\<[5]%
\>[5]{}\mathbf{do}\;{}\<[9]%
\>[9]{}\Varid{t}\leftarrow \Varid{rec}\;\Varid{e}{}\<[E]%
\\
\>[9]{}\Varid{return}\;(\Varid{tRef}\;\Varid{t}){}\<[E]%
\\
\>[3]{}\Varid{typeof_{Ref}}\;\Varid{rec}\;(\Conid{DeRef}\;\Varid{e})\mathrel{=}{}\<[E]%
\\
\>[3]{}\hsindent{2}{}\<[5]%
\>[5]{}\mathbf{do}\;{}\<[9]%
\>[9]{}\Varid{te}\leftarrow \Varid{rec}\;\Varid{e}{}\<[E]%
\\
\>[9]{}\Varid{maybe}\;\Varid{fail}\;\Varid{return}\;(\Varid{isTRef}\;\Varid{te}){}\<[E]%
\\
\>[3]{}\Varid{typeof_{Ref}}\;\Varid{rec}\;(\Conid{Assign}\;\Varid{e}_1\;\Varid{e}_2)\mathrel{=}{}\<[E]%
\\
\>[3]{}\hsindent{2}{}\<[5]%
\>[5]{}\mathbf{do}\;{}\<[9]%
\>[9]{}\Varid{t}_{1}\leftarrow \Varid{rec}\;\Varid{e}_1{}\<[E]%
\\
\>[9]{}\mathbf{case}\;\Varid{isTRef}\;\Varid{t}_{1}\;\mathbf{of}{}\<[E]%
\\
\>[9]{}\hsindent{3}{}\<[12]%
\>[12]{}\Conid{Nothing}\to \Varid{fail}{}\<[E]%
\\
\>[9]{}\hsindent{3}{}\<[12]%
\>[12]{}\Conid{Just}\;\Varid{t}\to \mathbf{do}\;{}\<[26]%
\>[26]{}\Varid{t}_{2}\leftarrow \Varid{rec}\;\Varid{e}_2{}\<[E]%
\\
\>[26]{}\mathbf{if}\;(\Varid{t}\equiv \Varid{t}_{2}){}\<[E]%
\\
\>[26]{}\hsindent{3}{}\<[29]%
\>[29]{}\mathbf{then}\;\Varid{return}\;\Varid{tUnit}{}\<[E]%
\\
\>[26]{}\hsindent{3}{}\<[29]%
\>[29]{}\mathbf{else}\;\Varid{fail}{}\<[E]%
\ColumnHook
\end{hscode}\resethooks
\end{minipage}

&

\begin{minipage}[t]{0.43\linewidth}
\vspace{-.05cm}

\hspace{.65cm}\ruleline{\scriptsize Monadic evaluation algebra}\begin{hscode}\SaveRestoreHook
\column{B}{@{}>{\hspre}l<{\hspost}@{}}%
\column{3}{@{}>{\hspre}l<{\hspost}@{}}%
\column{5}{@{}>{\hspre}l<{\hspost}@{}}%
\column{9}{@{}>{\hspre}l<{\hspost}@{}}%
\column{12}{@{}>{\hspre}l<{\hspost}@{}}%
\column{13}{@{}>{\hspre}l<{\hspost}@{}}%
\column{14}{@{}>{\hspre}l<{\hspost}@{}}%
\column{21}{@{}>{\hspre}l<{\hspost}@{}}%
\column{28}{@{}>{\hspre}l<{\hspost}@{}}%
\column{E}{@{}>{\hspre}l<{\hspost}@{}}%
\>[3]{}\Varid{eval_{Ref}}\mathbin{::}\mathbb{S}_{\scalebox{0.6}{M}}\;\Conid{Store}\;\Varid{m}\Rightarrow \Varid{Algebra}_M\;\Varid{Ref}_F\;(\Varid{m}\;\Conid{Value}){}\<[E]%
\\
\>[3]{}\Varid{eval_{Ref}}\;\Varid{rec}\;(\Conid{Ref}\;\Varid{e})\mathrel{=}{}\<[E]%
\\
\>[3]{}\hsindent{2}{}\<[5]%
\>[5]{}\mathbf{do}\;{}\<[9]%
\>[9]{}\Varid{v}{}\<[14]%
\>[14]{}\leftarrow \Varid{rec}\;\Varid{e}{}\<[E]%
\\
\>[9]{}\Varid{env}{}\<[14]%
\>[14]{}\leftarrow \Varid{get}{}\<[E]%
\\
\>[9]{}\Varid{put}\;(\Varid{v}\mathbin{:}\Varid{env}){}\<[E]%
\\
\>[9]{}\Varid{return}\;(\Varid{loc}\;(\Varid{length}\;\Varid{env})){}\<[E]%
\\
\>[3]{}\Varid{eval_{Ref}}\;\Varid{rec}\;(\Conid{DeRef}\;\Varid{e})\mathrel{=}{}\<[E]%
\\
\>[3]{}\hsindent{2}{}\<[5]%
\>[5]{}\mathbf{do}\;{}\<[9]%
\>[9]{}\Varid{v}{}\<[14]%
\>[14]{}\leftarrow \Varid{rec}\;\Varid{e}{}\<[E]%
\\
\>[9]{}\Varid{env}{}\<[14]%
\>[14]{}\leftarrow \Varid{get}{}\<[E]%
\\
\>[9]{}\mathbf{case}\;\Varid{isLoc}\;\Varid{v}\;\mathbf{of}{}\<[E]%
\\
\>[9]{}\hsindent{4}{}\<[13]%
\>[13]{}\Conid{Nothing}\to \Varid{return}\;\Varid{stuck}{}\<[E]%
\\
\>[9]{}\hsindent{4}{}\<[13]%
\>[13]{}\Conid{Just}\;\Varid{n}\to \Varid{return}\;(\Varid{maybe}\;\Varid{stuck}\;\Varid{id}\;(\Varid{fetch}\;\Varid{n}\;\Varid{env})){}\<[E]%
\\
\>[3]{}\Varid{eval_{Ref}}\;\Varid{rec}\;(\Conid{Assign}\;\Varid{e}_1\;\Varid{e}_2)\mathrel{=}{}\<[E]%
\\
\>[3]{}\hsindent{2}{}\<[5]%
\>[5]{}\mathbf{do}\;{}\<[9]%
\>[9]{}\Varid{v}{}\<[14]%
\>[14]{}\leftarrow \Varid{rec}\;\Varid{e}_1{}\<[E]%
\\
\>[9]{}\Varid{env}{}\<[14]%
\>[14]{}\leftarrow \Varid{get}{}\<[E]%
\\
\>[9]{}\mathbf{case}\;\Varid{isLoc}\;\Varid{v}\;\mathbf{of}{}\<[E]%
\\
\>[9]{}\hsindent{3}{}\<[12]%
\>[12]{}\Conid{Nothing}{}\<[21]%
\>[21]{}\to \Varid{return}\;\Varid{stuck}{}\<[E]%
\\
\>[9]{}\hsindent{3}{}\<[12]%
\>[12]{}\Conid{Just}\;\Varid{n}{}\<[21]%
\>[21]{}\to \mathbf{do}\;{}\<[28]%
\>[28]{}\Varid{v}_2\leftarrow \Varid{rec}\;\Varid{e}_2{}\<[E]%
\\
\>[28]{}\Varid{put}\;(\Varid{replace}\;\Varid{n}\;\Varid{v}_2\;\Varid{env}){}\<[E]%
\\
\>[28]{}\Varid{return}\;\Varid{unit}{}\<[E]%
\ColumnHook
\end{hscode}\resethooks
\end{minipage}\\
\hline
\end{tabular}
\end{center}
\vspace{-.2cm}
\caption{Syntax, type, and semantic function definitions for references.}
\label{fig:references}
\vspace{-.3cm}
\end{figure*}
