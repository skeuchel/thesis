\usepackage{ifdraft}

% If you ever would like to set up two different versions from the same code, you should look at the packages listed here:
% https://www.ctan.org/topic/cond-comp
%
% If you want to make a list of abbreviations, this blog explains how to do this:
% http://texblog.org/2014/01/15/glossary-and-list-of-acronyms-with-latex/

\ifdraft{
\usepackage{draftwatermark}
\SetWatermarkScale{2.0}
\SetWatermarkLightness{0.9}
}{}

% Warn about old packages
\RequirePackage[l2tabu, orthodox]{nag}

% English and Dutch, main language last.
\usepackage[dutch,english]{babel}

% Figures (titlepage among others)
\usepackage[square]{natbib}
\usepackage[dvipsnames,usenames]{color}
\usepackage{pgfplots}
\usepackage{graphicx}
\usepackage[all,cmtip]{xy}
\usepackage{mathpartir}
\usepackage{xcolor}
\usepackage{comment}
\usepackage{float}
\usepackage{xspace}
\usepackage{booktabs}
\usepackage{csquotes}
\usepackage{textgreek}
\usepackage{tikz}
\usepackage[normalem]{ulem}
\usepackage{colortbl}
\usepackage{afterpage}

% page dimensions
% taken from Michael's book, the name of this page format is StudienPartitur
% Karel also has 17cm x 24cm and says this is a B5. The print shop he used was University Press.
% Show the page margins during the writing process using the showframe option.
%
% I set the marginpar myself so that the todos appear nicely. This does NOT affect the textwidth, which is extremely important for me.
%
% For the crop marks on the draft version: see http://tex.stackexchange.com/questions/211248/problem-with-cropmarks-on-geometry-package
% However, you need to centre smaller page on the larger page for not cutting away text on the back of the page.
% This can be done using layouthoffset and layoutvoffset.
% Also see: http://tex.stackexchange.com/questions/211248/problem-with-cropmarks-on-geometry-package
% You have printed both portrait and landscape pages satisfactorily and have compared margins with other booklets.
\ifdraft{
  \usepackage[paper=a4paper,layoutwidth=170mm,layoutheight=240mm, marginparwidth=1in,
  showframe,showcrop,layouthoffset=20mm,layoutvoffset=28.5mm]{geometry}
}{
\usepackage[paperwidth=170mm,paperheight=240mm, marginparwidth=1in]{geometry}
}

% I want more space between the lines of the title.
% This package provides \setstretch/2.
\usepackage{setspace}

% Nice todo notes
% After geometry I guess, so that it is aware of the paper size
\ifdraft{
\usepackage{todonotes}
}{}

% Nice chapter headers, seven IKEA styles to choose from.
\usepackage[Lenny]{fncychap}

% Command \layout to insert a page containing layout with all the dimensions
\ifdraft{
\usepackage{layout}
}{}

% Automatically adds the bibliography and/or the index and/or the contents, etc., to the Table of Contents listing.
%
% I don't want the ToC in the ToC.
\usepackage[nottoc]{tocbibind}

% Usepackage caption
%
% captionof command
\usepackage[bf,small,hang]{caption}
\usepackage{subcaption}

% URL's
% Provided by the hyperref package.
%\usepackage{url}

% Popular mathematical packages
\usepackage{amsmath}
\usepackage{bm}
\usepackage{amssymb}
\usepackage{amsthm}
\usepackage{amsfonts}
\usepackage{mathtools}

\usepackage{relsize}

% At the end of each chapter, there is a reference to the article it is based on.
% Tom wants those in full.
% I tried the bibentry package, but it did not work.
% The biblatex package seems to be a completely different system from bibtex so I don't want to go there.

% Allows to turn certain pages into landscape
% The previous page may not be filled completely, so I provide my own environments for this
\usepackage{pdflscape}
\usepackage{afterpage}

% Floating landscape table on a page by its own where the previous page is filled completely.
% I tried defining an environment for this but just got nowhere...
%
% See http://tex.stackexchange.com/questions/19017/how-to-place-a-table-on-a-new-page-with-landscape-orientation-without-clearing-t
%
% Example usage:
%
%

% Geometry and landscape do not go well together: the margins are not adapted.
% Fix from http://tex.stackexchange.com/questions/115908/geometry-showframe-landscape
\makeatletter
\newcommand*{\gmshow@textheight}{\textheight}
\newdimen\gmshow@@textheight
\g@addto@macro\landscape{%
  \gmshow@@textheight=\hsize
  \renewcommand*{\gmshow@textheight}{\gmshow@@textheight}%
}
\def\Gm@vrule{%
  \vrule width 0.2pt height\gmshow@textheight depth\z@
}%
\makeatother

% Indexing, I haven't looked at Xindy yet
\usepackage{makeidx}
\makeindex

% I noticed that some figures are really far away, so let's refer to them
% intelligently using \cref (synonym: \vref). \Cref at the start of a
% sentence. See cleverref below.
\usepackage{varioref}

% Normally hyperref should be loaded as the last package, but there are exceptions
%
% To color the hyperlinks use the colorlinks option, but this may be distracting
\usepackage{hyperref}

% Varioref and hyperref do not go together (figure 4.7 f:ex), but loading this
% as the last package seems to fix it.
% Tom eist het gebruik van de capitalise optie.
\usepackage[capitalise]{cleveref}

% START visual index
%
% Source: http://tex.stackexchange.com/questions/119880/show-current-chapter-number-on-each-page-margin
% \usepackage[contents={},opacity=1,scale=1,color=black]{background}
% \usepackage{tikzpagenodes}
% \usepackage{totcount}
% \usetikzlibrary{calc}
%
% \newif\ifMaterial
%
% \newlength\LabelSize
% \setlength\LabelSize{2.5cm}
%
% \AtBeginDocument{%
% \regtotcounter{chapter}
% \setlength\LabelSize{\dimexpr\textheight/\totvalue{chapter}\relax}
% \ifdim\LabelSize>2.5cm\relax
%   \global\setlength\LabelSize{2.5cm}
% \fi
% }
%
% \newcommand\AddLabels{%
% \Materialtrue%
% \AddEverypageHook{%
% \ifMaterial%
% \ifodd\value{page} %
%  \backgroundsetup{
%   angle=90,
%   position={current page.east|-current page text area.north  east},
%   vshift=15pt,
%   hshift=-\thechapter*\LabelSize,
%   contents={%
%   \tikz\node[fill=gray!30,anchor=west,text width=\LabelSize,
%     align=center,text height=15pt,text depth=10pt,font=\large\sffamily] {\thechapter};
%   }%
%  }
%  \else
%  \backgroundsetup{
%   angle=90,
%   position={current page.west|-current page text area.north west},
%   vshift=-15pt,
%   hshift=-\thechapter*\LabelSize,
%   contents={%
%   \tikz\node[fill=gray!30,anchor=west,text width=\LabelSize,
%     align=center,text height=15pt,text depth=10pt,font=\large\sffamily] {\thechapter};
%   }%
%  }
%  \fi
%  \BgMaterial%
% \else\relax\fi}%
% }
%
% \newcommand\RemoveLabels{\Materialfalse}
%
% \AddLabels

% END visual index

% Make hyper-references back from bibliography to citation.
% Benoit: do I want this in the production version?
%--------------------------------------------------
% ... but for writing the document, this is probably useful.
%

% For an explanation of the options:
% http://ftp.snt.utwente.nl/pub/software/tex/macros/latex/contrib/hyperref/backref.pdf
% With option hyperpageref, the page numbers are hyperlinked.
\usepackage[hyperpageref]{backref}

% I don't like the style of the hyperlinked page numbers, so I redefine the
% \backref macro in terms of its old definition, see
% http://tex.stackexchange.com/questions/101693/how-to-do-a-recursive-renewcommand
%
% A nice style even for a production version, found on
% http://tex.stackexchange.com/questions/173046/modifying-backreference-page-numbers-in-references
\renewcommand\backrefxxx[3]{%
  \hyperlink{page.#1}{$\uparrow$#1}%
}

% Intended to check references in a document, looking for numbered but
% unlabelled equations, for labels which are not used in the text, for unused
% bibliography references.
%
% Note that refcheck should be loaded after the ams packages and the hyperref
% package.
% http://tex.stackexchange.com/questions/25742/how-can-i-get-latex-to-warn-about-unreferenced-figures-and-unused-labels-in-gene
\ifdraft{
\usepackage{refcheck}
}{}

% Command for the reference to the sources-of-a-chapter section or paragraph at
% the end of each chapter. I want this in the ToC, but I want an unnumbered
% section.
\newcommand{\chapterSource}{\section*{References} \addcontentsline{toc}{section}{References}}

% Define a custom indexing command so that it prints words added to the index in
% blue.  It used to be underlining, but this influences to linebreaks.  This is
% useful during the writing process, but you may want to change the definition
% for the final version.  Argument 1 can be a plural, argument 2 will be added
% to the index.  If argument 1 == argument 2, use \pindex{argument1}.
\ifdraft{
\newcommand{\myIndex}[2]{\textcolor{blue}{#1}\index{#2}}
}{
\newcommand{\myIndex}[2]{#1\index{#2}}
}

% Index a word without printing it in the final version. For the development
% version, print it as a marginnote that is not in the list of todos.
\ifdraft{
\newcommand{\npindex}[1]{\index{#1}\todo[color=yellow,nolist]{\tiny{Idx: #1}}}
}{
\newcommand{\npindex}[1]{\index{#1}}
}

% Print a word and index it too.
\newcommand{\pindex}[1]{\myIndex{#1}{#1}}

% Commands for making todo notes. \todo/1 is also available.
%\ifdraft{
\newcommand{\sk}[2]{\marginpar{\textcolor{green}{SK: #2}}\textcolor{green}{#1}}
\newcommand{\steven}[1]{\marginpar{\textcolor{green}{SK}}\textcolor{green}{#1}}
\newcommand{\tom}[1]{\marginpar{\textcolor{green}{TOM}}\textcolor{green}{#1}}
\newcommand{\scw}[1]{\marginpar{\textcolor{green}{SCW}}\textcolor{green}{#1}}
\newcommand{\new}[1]{\marginpar{\textcolor{blue}{NEW}}\textcolor{blue}{#1}}
\newcommand{\move}[1]{\marginpar{\textcolor{red}{MOVE}}\textcolor{red}{#1}}
%}{
%\newcommand{\sk}[1]{}
%\newcommand{\steven}[1]{}
%\newcommand{\tom}[1]{}
%\newcommand{\new}[1]{#1}
%}

% A command for making a structural note, which is different from a todo note.
% These notes won't appear in the list of todos.
\newcommand{\structureNote}[1]{}
%\newcommand{\structureNote}[1]{\todo[color=purple,nolist]{\tiny{#1}}}

% Newtheorem used in the definition of the example environment
% http://tex.stackexchange.com/questions/155710/understanding-the-arguments-in-newtheorem-e-g-newtheoremtheoremtheoremsec/155714#155714
% Met deze definitie zal de nummering altijd doorlopen over de verschillende
% hoofdstukken heen.  Dat is OK voor mij, maar een alternatief is bvb. de
% nummering te herstarten per hoofdstuk.
\newtheorem{exampleTheoremEnv}{Example}

% Environment for examples - It is useful to have your examples in this
% environment Because we may want to change the definition, we do not use a
% newtheorem directly
\newenvironment{example}
  {
     % Begin definitions
     \begin{exampleTheoremEnv}
  }
  {
     %End definitions
     \end{exampleTheoremEnv}
  }

\newtheorem{thm}{Theorem}
\newtheorem{lem}[thm]{Lemma}
\newtheorem{cor}[thm]{Corollary}
\newtheorem{defn}[thm]{Definition}
%% \newtheorem{example}[thm]{Example}

% numbervars is a function. For its argument, one can be sure to be in math mode.
\DeclareMathOperator{\numbervarsOperator}{numbervars}
\newcommand{\numbervars}[1]{\ensuremath{\numbervarsOperator(#1)}}

% Notation for the unique variables produced by the numbervars operation
% Argument is the number of the variable. F.e. \numberedVariable{1} produces a nice V_1 in a calligraphic font.
\newcommand{\numberedVariable}[1]{\ensuremath{\mathcal{V}_{#1}}}

% Powerset
%\DeclareMathOperator{\powerset}{\mathcal{P}}
% A subset of the powerset used in my answer subsumption chapter.
\DeclareMathOperator{\Card}{Card}

% Larget square subset equals for use over a set
% This doesn't seem to exists, or at least I couldn't find it quickly
\newcommand{\bigsqsubseteq}{\mathlarger{\sqsubseteq}}

\DeclareMathOperator{\ubs}{\textrm{ubs}}
\DeclareMathOperator{\Maybe}{Maybe}

\ifdraft{
\newcommand{\revisionBoundary}{\textcolor{red}{\hrulefill\quad {\Large revision boundary}\quad \hrulefill}}
}{
\newcommand{\revisionBoundary}{}
}

\definecolor{gray}{gray}{0.8}
\definecolor{dark-gray}{gray}{0.6}
\definecolor{light-gray}{gray}{0.9}
